\chapter{Đại số Boolean}

Boolean (hay luận lý) chỉ giá trị đúng hoặc sai của mệnh đề nào đó. 
Theo cách hiểu cơ bản, boolean gồm 2 giá trị 0 hoặc 1 (sai hoặc đúng).

\section{Hàm Boolean}

Hàm boolean $f$ đối với các biến $x_1, x_2, \ldots, x_n$ là hàm số nhận giá trị trong $\{0, 1\}^n$ và trả về giá trị thuộc $\{0, 1\}$.

Nghĩa là $f: \{0,1\}^n \mapsto \{0, 1\}$

\section{Các loại hàm boolean}

\begin{definition}[Đa thức Zhegalkin]
    Với hàm bool $f(x_1, x_2, \ldots, x_n)$, đa thức Zhegalkin tương ứng với hàm bool đó là cách biểu diễn
    đa thức đó dưới dạng tổng các tích như sau
    \[f(x_1, x_2, \ldots, x_n) = a_0 \oplus a_1 x_1 \oplus a_2 x_2 \oplus a_3 x_1 x_2 \oplus \ldots \oplus a_k x_1 x_2 \ldots x_n\]
    với $a_i \in \{0, 1\}$. Ta thấy rằng có $n$ biến, do đó có $2^n$ hệ số $a_i$.
\end{definition}

\begin{example}
    Cho hàm bool $f(x, y) = x \vee y$. Ta có bảng chân trị sau
    \begin{table}[ht]
        \centering
        \begin{tabular}{c c c}
            $x$ & $y$ & $f(x, y)$ \\
            0 & 0 & 0 \\
            0 & 1 & 1 \\
            1 & 0 & 1 \\
            1 & 1 & 1 \\
        \end{tabular}
    \end{table}
    Bảng chân trị này tương đương với đa thức Zhegalkin
    \[f(x, y) = x \oplus y \oplus xy\]
\end{example}
