\section{Cosets (chương 6)}

11. Gọi $H$ là subgroup của nhóm $G$ và giả sử $g_1, g_2 \in G$. Chứng minh các mệnh đề sau là tương đương:

\begin{itemize}
    \item[(a)] $g_1 H = g_2 H$
    \item[(b)] $H g_1^{-1} = H g_2^{-1}$
    \item[(c)] $g_1 H \subseteq g_2 H$
    \item[(d)] $g_2 \in g_1 H$
    \item[(e)] $g_1^{-1} g_2 \in H$ 
\end{itemize}

\begin{proof}
    Từ (a) ra (b): Ta đã biết các coset là rời nhau hoặc trùng nhau, do đó với mọi $g_1 h \in g_1 H$, tồn tại $g_2 h' \in g_2 H$ mà $g_1 h = g_2 h'$. Suy ra $(g_1 h)^{-1} = (g_2 h')^{-1}$ hay $h^{-1} g_1^{-1} = h'^{_1} g_2^{-1}$ (đpcm)

    Từ (a) ra (c): Hiển nhiên

    Từ (a) ra (d): Với mọi $g_1 h \in g_1 H$, tồn tại $g_2 h' \in g_2 H$ sao cho $g_1 h = g_2 h'$, hay $g_2 = g_1 h h'^{-1}$, đặt $h'' = h h'^{-1}$ thì $h'' \in H$ ($H$ là nhóm con) nên $g_1 h'' \in g_1 H$. Suy ra $g_2 \in g_1 H$

    Từ (a) ra (e): Tương tự, ta có $g_1 h = g_2 h'$, suy ra $h h'^{-1}= g_1^{-1} g_2 \in H$

\end{proof}


16. Nếu $g h g^{-1} \in H$ với mọi $g \in G$ và $h \in H$, chứng minh rằng right coset trùng với left coset

\begin{proof}
    Do $g h g^{-1} \in H$ nên tồn tại $h' \in H$ sao cho $g h g^{-1} = h'$. Tương đương $g h = h' g$ với mọi $h \in H$ nên $g H = H g$. Điều này đúng với mọi $g \in G$ nên các right coset trùng left coset.
\end{proof}

17. Giả sử $[G:H]=2$. Chứng minh rằng nếu $a, b$ không thuộc $H$ thì $ab \in H$.

\begin{proof}
    Ta biết rằng 2 coset ứng với 2 phần tử $g_1, g_2$ bất kì là trùng nhau hoặc rời nhau.

    Do đó với $eH = H$, ta suy ra 2 coset của $G$ là $H$ và $G \backslash H$.

    Vì $a, b \not\in H$ nên coset của chúng trùng nhau. Và nghịch đảo của $a$ cũng nằm trong $G \backslash H$ vì nếu nghịch đảo của $a$ nằm trong $H$ thì $a$ cũng phải nằm trong $H$.

    Suy ra $a^{-1} H = b H$. Nghĩa là tồn tại 2 phần tử $h_1, h_2 \in H$ sao cho $a^{-1} h_1 = b h_2$, tương đương $h_1 h_2^{-1} = a b \in H$ (đpcm).
\end{proof}

21. Gọi $G$ là cyclic group với order $n$. Chứng minh rằng có đúng $\phi(n)$ phần tử sinh của $G$

\begin{proof}
    Gọi $g$ là một phần tử sinh của $G$. Khi đó $g$ sinh ra tất cả phần tử trong $G$, hay nói cách khác các phần tử trong $G$ có dạng $g^i$ với $0 \leq i < n$.

    Như vậy một phần tử $h = g^i$ cũng là phần tử sinh của $G$ khi và chỉ khi $\gcd(i, n) = 1$ và có $\phi(n)$ số $i$ như vậy (đpcm).

\end{proof}