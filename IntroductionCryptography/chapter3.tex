\section{Chapter 3}

3.4. \textit{Euler's phi function $\phi(N)$} is the function defined by \[\phi(N) = | \{0 \leq k < N: \gcd(k, N)=1\} |\]
		 $\phi(p)=p-1$.

		 Consider the set $\{ai_1, ai_2, \ldots, ai_{\phi(N)}\}$ is the set of numbers which are coprime with $N$, which means $\gcd(ai_j, N)=1$. We prove that those elements are distinct.
		
		Suppose that there are $aj$ and $ak$, satisfying $aj \equiv ak \pmod{N}$
		
		Because $\gcd(a, N)=1 \Rightarrow j \equiv k \pmod{N}$. So every element is distinct. 
		
		Moreover, if $ai_j \equiv j_k \pmod{N}$, which means $j_k \neq 0$, so the set $\{ai_1, \ldots, ai_{\phi(N)}\}$ is a permutation of the set $\{i_1, \ldots, i_{\phi(N)}\}$
		\[ai_1 \times ai_2 \times \cdots \times ai_{\phi(N)} \equiv i_1 \times i_2 \times \cdots \times i_{\phi(N)} \pmod{N}\]
		
		Therefore $a^{\phi(N)} \equiv 1 \pmod{N}$
	

3.5. Properties of Euler's phi function
		 If $p$ and $q$ are distinct primes, how is $\phi(pq)$ related to $\phi(p)$ and $\phi(q)$? \\ We consider numbers from 1 to $pq$, there are $pq$ elements \\ Notice that $iq = jq$ if and only if $i=q$ and $j=p$ because $p$ and $q$ are distinct primes \\ Next, we subtract the number of divisors having factor $p$, there are $q$ elements ($1 \times p, 2 \times p, \cdots, q \times p$) \\ Next, we subtract the number of divisors having factor $q$, there are $p$ elements ($1 \times q$, $2 \times q, \cdots, p \times q$) \\ Here we get $pq - p - q$ elements, but remember that we have subtracted element $pq$ twice, so we need to add 1 \\ $\Rightarrow \phi(pq) = pq - p - q + 1 = (p-1)(q-1) = \phi(p)\phi(q)$
		 If $p$ is prime, what is the value of $\phi(p^2)$? How about $\phi(p^j)$? \\ From 1 to $p^j$ there are $p^j$ elements, we subtract the number of divisors having factor $p$, those are $\{1p, 2p, \cdots, p^{j-1}p\} \Rightarrow p^{j-1}$ numbers \\ $\Rightarrow \phi(p^j) = p^j - p^{j-1}$
		 We write numbers from 1 to $mn$ as matrix $m$ rows and $n$ columns
		\begin{center}
			\begin{tabular}{c c c c}
				$0m+1$ & $1m+1$ & $\cdots$ & $(n-1)m+1$ \\ 
				$0m+2$ & $1m+2$ & $\cdots$ & $(n-1)m+2$ \\ 
				$\cdots$ & $\cdots$ & $\cdots$ & $\cdots$ \\ 
				$0m + m-1$ & $1m + m-1$ & $\cdots$ & $(n-1)m + m - 1$ \\
				$0m + m$ & $1 m + m$ & $\cdots$ & $(n-1)m + m$
			\end{tabular}
		\end{center}
		With number $r$ that satisfies $\gcd(r,m)=1$, we get $gcd(km+r,r)=1$ ($k=\overline{0, n-1}$). Here $km+r$ is all numbers on $r$-th row, which means there are $\phi(m)$ rows, whose elements coprime with $m$ \\ On those $\phi(m)$ rows, each row has $\phi(n)$ elements that coprime with $n$. Hence $\phi(m)\phi(n)=\phi(mn)$
		 From (b) we get $\phi(p_i) = p_i - 1$
		\begin{align*}
			\Rightarrow \phi(N) & = \phi(p_1)\phi(p_2)\cdots\phi(p_r) \\ & = (p_1 - 1)(p_2 -1)\cdots(p_r-1) \\ & = N\prod_{i=1}^{r}\Bigg(1-\frac{1}{p_i}\Bigg)
		\end{align*}
	


3.6. Let $N$, $c$, and $e$ be positive integers satisfying the conditions $\gcd(N,c)=1$ and $\gcd(e,\phi(N))=1$
		 Explain how to solve the congruence \[x^e \equiv c \pmod{N}\] assuming that you know the value of $\phi(N)$ \\ Because of $\gcd(e, \phi(N)) = 1$, we can find an integers $d$ satisfying that $ed \equiv 1 \pmod{\phi(N)}$ (using Extended Euclidean Algorithm) \\ $\Rightarrow ed = k\phi(N) + 1$ with $k \in \mathbb{Z}$ \\ And because of $\gcd(N, c)=1 \Rightarrow \gcd(N,x)=1$, and \[c^d = \Big(x^e\Big)^d = x^{ed} = x^{k\phi(N) + 1} = (x^k)^{\phi(N)}x\] and we have known that $(x^k)^{\phi(N)} \equiv 1 \pmod{N}$ from Exercise 3.4. Therefore we get \[c^d \equiv x \pmod{N}\], we finish finding solution
	


3.11. Alice chooses two large primes $p$ and $q$ and she publishes $N=pq$. It is assumed that $N$ is hard to factor. Alice also chooses three random numbers $g$, $r_1$, and $r_2$ modulo $N$ and computes $$g_1 \equiv g^{r_1(p-1)} \pmod{N} \qquad \text{and} \qquad g_2 \equiv g^{r_2(q-1)} \pmod{N}$$ Her public key is the triple $(N, g_1, g_2)$ and her private key is the pair of primes $(p, q)$.
	
	Now Bob wants to send the message $m$ to Alice, where $m$ is a number modulo $N$. He chooses two random integers $s_1$ and $s_2$ modulo $N$ and computes $$c_1 \equiv mg_1^{s_1} \pmod{N} \qquad \text{and} \qquad c_2 \equiv mg_2^{s_2} \pmod{N}$$ Bob sends the ciphertext $(c_1, c_2)$ to Alice.
	
	Decryption is extreamly fast and essy. Alice uses the Chinese remainder theorem to solve the pair of congruences \[x \equiv c_2 \pmod{p} \qquad \text{and} \qquad x \equiv c_2 \pmod{q}\]
		
	Prove that Alice's solution $x$ is equal to Bob's plaintext $m$ \\ First we have $c_1 \equiv mg_1^{s_1} \pmod{N} \equiv mg_1^{s_1} \pmod{p} \equiv m \pmod{p}$ \\ (because $g_1^{s_1} = (g_1^{s_1 r_1})^{(p-1)} \equiv 1 \pmod{p}$) \\ Similarly, we have $c_2 \equiv m \pmod{q}$ \\ The solution of congurences is \[x \equiv c_1 q q' + c_2 p p' \pmod N\] with $p p' + q q' = 1$ \\ $\Rightarrow x \equiv m p p' + m q q' \equiv m(p p' + q q') \equiv m \pmod N$
		 
	We have 
	\[g_1 \equiv g^{r_1 (p-1)} \pmod N \equiv g^{r_1 (p-1)} \pmod p \equiv 1 \pmod p\]
	
	$\Rightarrow p = \gcd(g_1-1, N)$. Similarly, $q = \gcd(g_2-1, N)$
	
	From here we have recovered private keys
	


3.13. Find $x$, $y$ such that: $xe_1 + ye_2=1=gcd(e_1,e_2)$ \\ $\Rightarrow m=c_1^x c_2^y = m^{e_1 x + e_2 y} = m \pmod{N}$

3.14. Because 3, 11 and 17 are primes number, $a \equiv a^3 \pmod 3$, $a \equiv a^{11} \pmod{11}$, $a \equiv a^{17} \pmod{17}$. We have system congruence
		\begin{align*}
			a & \equiv a^3 \pmod 3 \\ a & \equiv a^{11} \pmod{11} \\ a & \equiv a^{17} \pmod{17}
		\end{align*}
		Consider that $a^3 \equiv a \pmod 3$, $a^{3^2} \equiv a^3 \equiv a \pmod 3$, $\cdots$, $a^{3^i} \equiv a \pmod 3$. And $561 = 2 \cdot 3^5 + 2 \cdot 3^3 + 2 \cdot 3^2 + 3^1$, $a^{561} \equiv a^2 \cdot a^2 \cdot a^2 \cdot a \equiv a^9 \equiv a \pmod 3$.
		
		Similarly, $a^{561} \equiv a \pmod{11}$, $a^{561} \equiv a \pmod{17}$. From system congruence:
		\begin{align*}
			a^{561} & \equiv a \pmod 3 \\ a^{561} & \equiv a \pmod{11} \\ a^{561} & \equiv a \pmod{17}
		\end{align*}
		Using CRT, $a^{561} = (187 \cdot 1 \cdot a + 51 \cdot 8 \cdot a + 33 \cdot 16 \cdot a) \pmod{561} = a \pmod{561}$
		 Suppose that $n$ is even ($n \geq 4$), we have 
		\[(n-1)^{n-1} = (-1)^{n-1} = -1 \pmod n\]
		, but $a^{n-1} \equiv 1 \pmod n$ for all $a$, which is contrary. So $n$ must be odd.
		 Suppose that $n = p_1^{e_1} p_2^{e_2} \cdots p_r^{e_r}$ ($p_i$ is odd prime). Because $a^{p^{e-1} (p-1)} \equiv 1 \pmod{p^e}$ and $a^{n-1} \equiv 1 \pmod n$, we have $a^{n-1} \equiv 1 \pmod{p^e}$. \\ $\Rightarrow p^{e-1}(p-1) \mid (n-1) \Rightarrow p^{e-1} \mid (n-1)$, but $p^{e-1} \mid n$, which is contrary if $e \geq 2$. Hence $e$ must be 1.
		
		So $n = p_1 p_2 \cdots p_r$
		


3.37. \[\Big(a^{\frac{p-1}{2}}\Big)^2 \equiv a^{p-1} \equiv 1 \pmod p\]

$\Rightarrow \binom{a}{p} = \pm 1$

$\Rightarrow \Big(a^{\frac{p-1}{2}} - 1\Big)\Big(a^{\frac{p-1}{2}} + 1\Big) \equiv 0 \pmod p$

$\Rightarrow a^{\frac{p-1}{2}} \equiv \pm 1 \pmod p$
	 
If $a$ is quadratic residue, then $a \equiv b^2 \pmod p$ 

$\Rightarrow a^{\frac{p-1}{2}} \equiv (b^2)^{\frac{p-1}{2}} = b^{p-1} \equiv 1 \pmod p$

If $a^{\frac{p-1}{2}} \equiv 1 \pmod p$

Let $g$ be generator modulo $p$, then $a \equiv g^m \pmod p$

If $m$ is even $\Rightarrow a \equiv g^{2k} \pmod p \Rightarrow a^{\frac{p-1}{2}} \equiv 1 \pmod p$

If $m$ is odd $\Rightarrow a = g^{2k+1} \pmod p \Rightarrow a^{\frac{p-1}{2}} \equiv g^{(2k+1)\frac{p-1}{2}} \equiv g^{p-1} g^{\frac{p-1}{2}} \equiv g^{\frac{p-1}{2}}\not\equiv 1 \pmod p$, because $p-1$ is smallest number that $g^{p-1} \equiv 1 \pmod p$

From (a) and (b) $\binom{-1}{p} \equiv (-1)^{\frac{p-1}{2}} \pmod p$, if $p=4k+1 \Rightarrow (-1)^{\frac{p-1}{2}} \equiv (-1)^{2k} \equiv 1 \pmod p$ \\ If $p=4k+3 \Rightarrow (-1)^{\frac{p-1}{2}} \equiv (-1)^{2k+1} \equiv -1 \pmod p$

3.38. Prove that the three parts of the quadratic reciprocity theorem are equivalent to the following three concise formulas, where $p$ and $q$ are odd primes

		 $\Big(\frac{-1}{p}\Big) = (-1)^{\frac{p-1}{2}}$
		
		With $p \equiv 1 \pmod 4 \Rightarrow \Big(\frac{-1}{p}\Big) = 1 = (-1)^{\frac{p-1}{2}} \pmod p$ \\ Similarly with $p \equiv 3 \pmod 4$
		 $\Big(\frac{2}{p}\Big) = (-1)^{\frac{p^2-1}{8}}$
		
		First we need a lemma (\textbf{Gauss lemma}): suppose $p$ is an odd prime, and $a \in \mathbb{Z}$, $\gcd(a, p) = 1$. Consider the set
		\[a, 2a, 3a, \cdots, \frac{p-1}{2}a\]
		
		If $s$ of those residues are greater than $\frac{p}{2}$, then $\binom{a}{p} = (-1)^s$
		
		\underline{Proof of lemma}: Among smallest residues of 
		\[a, 2a, 3a, \cdots, \frac{p-1}{2}a\], suppose that 
		\[u_1, u_2, \cdots, u_s\] are residues greater than $\frac{p}{2}$, and 
		\[v_1, v_2, \cdots, v_t\] are residues smaller than $\frac{p}{2}$
		
		Because $\gcd(ja, p)=1 \forall j, 1 \leq j \leq \frac{p-1}{2}$, all $u_i, v_j \neq 0 \Leftrightarrow u_i, v_j \in \{1,2,\cdots,p-1\}$. We will prove that, the set 
		\[\{p-u_1, p-u_2, \cdots, p-u_s, v_1, v_2, \cdots, v_t\}\] 
		is a permutation of $\{1,2,\cdots,\frac{p-1}{2}\}$
		
		It is clear that there are no 2 numbers $u_i$ or 2 numbers $v_j$ simultaneously congruent modulo $p$. Because if $ma \equiv na \pmod p$ and $\gcd(a,p)=1$, then $m\equiv n \pmod p \Rightarrow$ contrast with $m,n \leq \frac{p-1}{2}$
		
		Similarly, we see that there are no numbers $p-u_i$ congruent with $v_j$, so $$\Rightarrow (p-u_1)(p-u_2)\cdots(p-u_s)v_1 v_2 \cdots v_t \equiv \Big(\frac{p-1}{2}\Big)! \pmod p$$
		
		On the other hand, 
		\[u_1,u_2,\cdots,u_s, v_1, v_2, \cdots, v_t\] 
		are smallest residues of 
		\[a, 2a, 3a, \cdots, \frac{p-1}{2}\], so 
		\[\Rightarrow u_1 u_2 \cdots u_s v_1 v_2 \cdots v_t \equiv a^{\frac{p-1}{2}} \Big(\frac{p-1}{2}\Big)! \pmod p\]
		
		So $(-1)^s a^{\frac{p-1}{2}} \Big(\frac{p-1}{2}\Big)! \equiv \Big(\frac{p-1}{2}\Big)! \pmod p$
		
		And because $\gcd(p, \Big(\frac{p-1}{2}\Big)!) = 1 \Rightarrow (-1)^s a^{\frac{p-1}{2}} \equiv 1 \pmod p$
		
		$\Rightarrow a^{\frac{p-1}{2}} \equiv (-1)^s \pmod p$ and $\binom{a}{p} = a^{\frac{p-1}{2}}$
		
		$\Rightarrow \binom{a}{p} = (-1)^s \pmod p$
		
		\underline{Return to problem}: using theorem above, we need to find the number of residues, which are greater than $\frac{p}{2}$ among $1 \cdot 2$, $2 \cdot 2$, $\cdots$, $\frac{p-1}{2} \cdot 2$. Therefore we only need to know which numbers are greater than $\frac{p}{2}$
		
		$\Rightarrow$ there are $s = \frac{p-1}{2} - \Big[\frac{p}{4}\Big] \Rightarrow \Big(\frac{2}{p}\Big) = (-1)^{\frac{p-1}{2} - \Big[\frac{p}{4}\Big] }$
		
		With $p \equiv 1, 3, 5, 7 \pmod 8$, we have 
		\[\frac{p-1}{2} - \Big[\frac{p}{4}\Big] \equiv \frac{p^2-1}{8} \pmod 2\] 
		
		\[\Rightarrow \Big(\frac{2}{p}\Big) = (-1)^{\frac{p^2-1}{8}}\]

		 $\Big(\frac{p}{q}\Big)\Big(\frac{q}{p}\Big) = (-1)^{\frac{p-1}{2} \cdot \frac{q-1}{2}}$
		
		We need a lemma: Suppose $p$ is an odd prime, $a$ is odd and $\gcd(a, p) = 1$, then $\Big(\frac{a}{p}\Big) = (-1)^{T(a, p)}$, with 
		\[T(a, p) = \sum_{j=1}^{\frac{p-1}{2}}\Big[\frac{ja}{p}\Big]\]
		
		\underline{Proof of lemma}: consider smallest residues of $a$, $2a$, $3a$, $\cdots$, $\frac{p-1}{2} \cdot a$. As Gauss's lemma, $u_1$, $u_2$, $\cdots$, $u_s$, $v_1$, $v_2$, $\cdots$, $v_t$ are residues greater and less than $\frac{p}{2}$ respectively. According to Euclidean divisor: $$ja = p \Big[\frac{ja}{p}\Big] + \text{remainder}$$, remainder is $u_i$ or $v_j$. We have such $\frac{p-1}{2}$ equations and add them together $$\Rightarrow \sum_{j=1}^{\frac{p-1}{2}}ja = \sum_{j=1}^{\frac{p-1}{2}}p\Big[\frac{ja}{p}\Big] + \sum_{i=1}^{s}u_i + \sum_{j=1}^{t}v_j$$
		As we pointed out in Gauss's lemma, the set $p-u_1$, $p-u_2$, $\cdots$, $p-u_s$, $v_1$, $v_2$, $\cdots$, $v_t$ is a permutation of the set 1, 2, $\cdots$, $\frac{p-1}{2}$
		$$\sum_{j=1}^{\frac{p-1}{2}}j = \sum_{i=1}^{s}(p-u_i) + \sum_{j=1}^{t}v_t = ps - \sum_{i=1}^{s}u_i + \sum_{j=1}^{t}v_t$$
		$$\Rightarrow \sum_{j=1}^{\frac{p-1}{2}}ja - \sum_{j=1}^{\frac{p-1}{2}}j = \sum_{j=1}^{\frac{p-1}{2}}p\Big[\frac{ja}{p}\Big] - ps + 2\sum_{i=1}^{s}u_i$$
		From formula of $T(a, p)$, $(a-1)\sum_{j=1}^{\frac{p-1}{2}}j = pT(a, p) - ps + 2 \sum_{i=1}^{s}u_i$
		
		Because $a$, $p$ are odd, $T(a, p) \equiv s \pmod 2$. From Gauss's lemma we finish.
		
		\underline{Return to problem}: Consider pairs $(x, y)$, where $1 \leq x \leq \frac{p-1}{2}$ and $1 \leq y \leq \frac{q-1}{2}$, there are $\frac{p-1}{2}\cdot\frac{q-1}{2}$ pairs. We divide those pairs into 2 groups, depending on the magnitude of $px$ and $qy$.
		
		Because $p$, $q$ are two different primes, $px \neq qy$, $\forall (x, y)$
		
		We consider pairs with $qx > py$. With every fixed element of $x$ ($1 \leq x \leq \frac{p-1}{2}$), exist $\Big[\frac{qx}{p}\Big]$ elements $y$ satisfying $1 \leq y \leq \frac{qx}{p}$. Therefore, there are $\sum_{i=1}^{\frac{p-1}{2}}\Big[\frac{iq}{p}\Big]$ pairs. When $qx < py$, similarly, there are $\sum_{j=1}^{\frac{q-1}{2}}\Big[\frac{jp}{q}\Big]$ pairs. Because there are $\frac{p-1}{2} \cdot \frac{q-1}{2}$ pairs, we have equation $$\sum_{i=1}^{\frac{p-1}{2}}\Big[\frac{iq}{p}\Big] + \sum_{j=1}^{\frac{q-1}{2}}\Big[\frac{jp}{q}\Big] = \frac{p-1}{2} \cdot \frac{q-1}{2}$$
		From definition of $T(p, q)$, we have result $$(-1)^{T(p, q) + T(q, p)} = (-1)^{\frac{p-1}{2} \cdot \frac{q-1}{2}}$$
	


3.39 Let $p$ be a prime satisfying $p \equiv 3 \pmod 4$.

		 Let $a$ be a quadratic residue modulo $p$. Prove that the number $$b \equiv a^{\frac{p+1}{4}} \pmod p$$ has the property that $b^2 \equiv a \pmod p$. (\textit{Hint}. Write $\frac{p+1}{2}$ as $1+\frac{p-1}{2}$ and use Exercise 3.37.) This gives an easy way to take square roots modulo $p$ for primes that are congruent to 3 modulo 4.
		
		 \begin{proof}
			Using Exercise 3.37, $a^{\frac{p-1}{2}} \equiv 1 \pmod p$ because $a$ is quadratic residue modulo $p$. Therefore 
			\[b^2 \equiv a^{\frac{p+1}{2}} \equiv a^{1+\frac{p-1}{2}} \equiv a \cdot a^{\frac{p-1}{2}} \equiv a \cdot 1 \equiv 1 \pmod p\]
		 \end{proof}


3.40. Let $p$ be and odd prime, let $g \in \mathbb{F}^{*}_p$ be a primitive root, and let $h \in \mathbb{F}^{*}_p$. Write $p - 1 = 2^sm$ with $m$ odd and $s \geq 1$, and write the binary expansion of $\log_g(h)$ as $$log_g(h) = \epsilon_0 + 2\epsilon_1 + 4\epsilon_2 + 8\epsilon_3 + \cdots \quad \text{with} \quad \epsilon_0, \epsilon_1, \cdots \in \{0, 1\}$$
	Give an algorithm that generalizes Example 3.69 and allows you to rapidly compute $\epsilon_0, \epsilon_1, \cdots, \epsilon_{s-1}$, thereby proving that the first $s$ bits of the discrete logarithm are insecure.
	
	\begin{algorithm}
		\caption{Algorithm to find $s$ least significant bits of $x$ in $g^x \equiv h \pmod p$}
		\begin{algorithmic}
			\Require{$g$, $h$, $p$ ($p-1=2^s m$)}
			\Ensure{$s$ least significant bits of $x: g^x \equiv h \pmod p$}
			
			Array $\epsilon_0, \epsilon_1, \cdots, \epsilon_{s-1}$\;
			\For{$i = 0, \ldots, s-1$}{
				\If{$h$ is quadratic residue}
					$\epsilon_i = 0$, $h = \sqrt{h} \pmod p$
				\ElsIf{$\epsilon_i = 1$}
					$h = \sqrt{g^{-1}h} \pmod p$
				\EndIf
			\EndFor
			}
		\end{algorithmic}
	\end{algorithm}

3.41 Let $p$ be a prime satisfying $p \equiv 1 \pmod 3$. We say that $a$ is a \textit{cubic residue modulo $p$} if $p \nmid a$ and there is an integer $c$ satisfying $a \equiv c^3 \pmod p$.
	
Let $a$ and $b$ be cubic residues modulo $p$. Prove that $ab$ is a cubic residue modulo $p$.

\begin{proof}
	$a \equiv x^3 \pmod p$, $y \equiv y^3 \pmod p$. Therefore $$ab \equiv x^3 y^3 = (xy)^3 \pmod p$$, which is cubic residue
	Give an example to show that (unlike the case with quadratic residues) it is possible for none of $a$, $b$ and $ab$ to the a cubic residue modulo $p$

Let $g$ be primitive root modulo $p$. Choose $a \equiv g^{3k+1} \pmod p$, $b \equiv g^{3k'+1} \pmod p$. Hence $ab \equiv g^{(3k+1)+(3k'+1)} \equiv g^{3(k+k')+2} \pmod p$, which is not cubic residue
	Let $g$ be a primitive root modulo $p$. Prove that $a$ is a cubic residue modulo $p$ if and only if $3 \mid \log_g(a)$, where $\log_g(a)$ is the discrete logarithm of $a$ to the base $g$.

\underline{Proof of sufficient condition}: If $a$ is a cubic residue modulo $p$, $3 \mid \log_g(a)$. Suppose $a \equiv c^3 \pmod 3$ and $c \equiv g^u = \pmod p$. Hence $a = g^3u \pmod p \Rightarrow 3 \mid \log_g(a)$

\underline{Proof of necessary condition}: If $3 | \log_g(a)$, $a$ is a cubic residue modulo $p$. This is obviously.
	Suppose instead that $p \equiv 2 \pmod 3$. Prove that for every integer $a$ there is an integerr $c$ satisfying $a \equiv c^3 \pmod p$. In other words, if $p \equiv 2 \pmod 3$, show that every number is a cube modulo $p$.

\underline{Return to problem}: Because $p \equiv 2 \pmod 3 \Rightarrow \gcd(p-1, 3) = 1$. Which means that exist element $d$ such that $3d \equiv 1 \pmod{p-1}$. Hence, equation $x^3 \equiv a \pmod p$ has solution $a^d = x \pmod p$. So every number is a cube modulo $p$.
	
\end{proof}
