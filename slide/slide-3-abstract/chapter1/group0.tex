\section{Lý thuyết nhóm}

\begin{frame}{Nhóm (Group)}
    Một nhóm (Group) là một tập hợp $G$ khác rỗng và một toán tử 2 ngôi trên $G$ là $\star$ thỏa mãn các tính chất:
    \begin{itemize}
        \item Tính đóng (Closure): với mọi $a, b \in G$, $a \star b \in G$. (Toán tử 2 ngôi trên $G$)
        \item Tính kết hợp (Associative Law): với mọi $a$, $b$, $c \in G$, $(a \star b) \star c = a \star (b \star c)$
        \item Tồn tại phần tử đơn vị (Identity Law):tồn tại  $e \in G$ sao cho $a \star e = e \star a = a$ với mọi $a \in G$
        \item Tồn tại phần tử nghịch đảo (Inverse Law): với mọi $a \in G$, tồn tại $a' \in G$ sao cho $a \star a' = a' \star a = e$
    \end{itemize}
\end{frame}

\begin{frame}{Ví dụ về nhóm}
    \textbf{Ví dụ 1}. Xét tập số nguyên $\mathbb{Z}$ và phép toán cộng ($+$) trên $\mathbb{Z}$.
    
    \begin{itemize}
        \item Ta nhận thấy rằng với mọi $a \in \mathbb{Z}$ thì $a + 0 = 0 + a = a$. Như vậy phần tử đơn vị của nhóm này là $0$.
        \item Thêm nữa, với mọi $a \in \mathbb{Z}$ thì $a + (-a) = (-a) + a = 0$. Do đó phần tử đối của phần tử $a$ bất kì là $-a$. Ví dụ phần tử đối của 3 là -3, của -10 là 10.
        \item Đối với tính kết hợp, do phép cộng số nguyên có tính kết hợp nên dễ dàng suy ra.
    \end{itemize}

    Như vậy tập hợp các số nguyên $\mathbb{Z}$ và phép cộng các số nguyên là tạo thành một nhóm.
\end{frame}

\begin{frame}{Ví dụ về nhóm}
    \textbf{Ví dụ 2}. Xét tập số hữu tỷ không chứa số 0 là $\mathbb{Q}^*$ và phép nhân $\times$ các số hữu tỉ.
    
    \begin{itemize}
        \item Phần tử đơn vị: với mọi $q \in \mathbb{Q}^*$ thì $q \times 1 = 1 \times q = q$. Như vậy phần tử đơn vị là 1.
        \item Phần tử nghịch đảo: với mọi $q \in \mathbb{Q}^*$, phần tử nghịch đảo là số $q'$ sao cho $q \times q' = q' \times q = 1$. Suy ra $q' = \frac{1}{q}$. Ví dụ nghịch đảo của 4 là $\frac{1}{4}$, nghịch đảo của $\frac{5}{3}$ là $\frac{3}{5}$.
        \item Tính kết hợp: Do tính chất kết hợphợp của phép nhân các số hữu tỉ nên dễ nhận thấy.
    \end{itemize}

    Như vậy, tập hợp các số hữu tỷ khác 0 $\mathbb{Q}^*$ và phép nhân các số hữu tỷ tạo thành một nhóm.
\end{frame}

\begin{frame}{Ví dụ về nhóm}
    \textbf{Ví dụ 3}. Xét tập hợp các ma trận $2\times 2$ $GL_2(\mathbb{R}) = \begin{pmatrix}a & b \\ c & d\end{pmatrix}$ có định thức khác 0 ($ad - bc \neq 0$) và phép nhân ma trận.
    \begin{itemize}
        \item Tính đóng: do $\det(AB) = \det(A) \det(B)$ nên do cả $\det(A)$ và $\det(B)$ đều khác 0 nên $\det(AB)$ cũng khác 0. Do đó ma trận tích $AB \in GL_2(\mathbb{R})$.
        \item Phần tử đơn vị: ma trận đơn vị $I = \begin{pmatrix}1 & 0 \\ 0 & 1\end{pmatrix}$ thỏa mãn $I A = A I = A$ với mọi ma trận $A \in GL_2(\mathbb{R})$.
        \item Phần tử nghịch đảo: do các ma trận trong tập hợp có định thức khác 0 nên luôn tồn tại nghịch đảo cho mọi ma trận $A \in GL_2(\mathbb{R})$.
        \item Tính kết hợp: theo tính chất phép nhân ma trận thì $(AB)C = A(BC)$ với mọi ma trận $A, B, C \in GL_2(\mathbb{R})$
    \end{itemize}
    
    Như vậy tập hợp các ma trận $2 \times 2$ với định thức khác 0 $GL_2(\mathbb{R})$ cùng phép nhân các ma trận tạo thành một nhóm (\textbf{General Linear Group}).
\end{frame}

\begin{frame}{Đặc trưng của nhóm}
    \textbf{Định nghĩa}. \textbf{Order} (Cấp) của nhóm là số lượng phần tử của $G$ và được ký hiệu là $\# G$.
    
    Trong các ví dụ trên thì $\# G = \infty$, nghĩa là nhóm có vô hạn phần tử. Các ví dụ sau sẽ xét về các nhóm hữu hạn phần tử, hay còn gọi là nhóm hữu hạn.
\end{frame}

\begin{frame}{Permutation Group (Nhóm hoán vị)}
    Cho số tự nhiên $n$ và gọi $P = \{1, 2, \cdots, n\}$. Tập hợp tất cả hoán vị của $P$ và toán tử sau đây tạo thành một nhóm: với mọi hoán vị $x, y$ thì $x \star y$ là vị trí của $x$ theo $y$.
    
    Ví dụ, với $n=4$. Xét 2 hoán vị $x = (1, 4, 2, 3)$ và $y = (3, 2, 4, 1)$. Khi đó gọi $z = x \star y$ thì: 
    
    $z_1 = x_{y_1} = x_3 = 2,\ z_2 = x_{y_2} = x_2 = 4,\ z_3 = x_{y_3} = x_4 = 3,\ z_4 = x_{y_4} = x_1 = 1$
    
    Như vậy $z = x \star y = (2, 4, 3, 1)$.
    
    Ở đây ta cũng có thể viết dưới dạng ánh xạ thay vì dùng chỉ số của dãy số. Nghĩa là $x(1) = 1,\ x(2) = 4,\ x(3)=2,\ x(4)=3$ và $y(1)=3,\ y(2)=2,\ y(3)=4,\ y(4)=1$. 
    
    Khi đó $\star$ là hợp của 2 ánh xạ $z(i) = x(y(i))$.
\end{frame}

\begin{frame}{Permutation Group (Nhóm hoán vị)}
    
\end{frame}
\begin{frame}{Dihedral Group (Nhóm Dihedral)}
    
\end{frame}