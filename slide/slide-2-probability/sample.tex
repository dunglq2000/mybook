%----------------------------------------------------------------------------------------
%	PACKAGES AND THEMES
%----------------------------------------------------------------------------------------
\documentclass[aspectratio=169,xcolor=dvipsnames,serif]{beamer}
\usetheme{SimplePlus}

\usepackage{hyperref}
\usepackage{graphicx} % Allows including images
\usepackage{booktabs} % Allows the use of \toprule, \midrule and \bottomrule in tables
\usepackage[utf8]{vietnam}
\usepackage{amsmath,bm}

%----------------------------------------------------------------------------------------
%	TITLE PAGE
%----------------------------------------------------------------------------------------

\title[short title]{Xác suất thống kê} % The short title appears at the bottom of every slide, the full title is only on the title page
\subtitle{Probability}

\author{Lê Quốc Dũng}

\institute[MPEI] % Your institution as it will appear on the bottom of every slide, may be shorthand to save space
{
    Moscow Power Engineering Institute % Your institution for the title page
}
\date{\today} % Date, can be changed to a custom date


%----------------------------------------------------------------------------------------
%	PRESENTATION SLIDES
%----------------------------------------------------------------------------------------

\begin{document}

\begin{frame}
    % Print the title page as the first slide
    \titlepage
\end{frame}

\begin{frame}{Mục lục}
    % Throughout your presentation, if you choose to use \section{} and \subsection{} commands, these will automatically be printed on this slide as an overview of your presentation
    \tableofcontents
\end{frame}

%------------------------------------------------
\section{Xác suất cổ điển}

\begin{frame}{Định nghĩa cổ điển của xác suất}
    Định nghĩa thống kê của xác suất:
    
    Trong một phép thử có $n$ khả năng có thể xảy ra. Xét một biến cố $A$ xảy ra khi thực hiện phép thử có $k$ khả năng xảy ra. Khi đó xác suất của biến cố $A$ ký hiệu là $P(A)$ và được tính \[P(A) = \frac{k}{n}\]
    
    Dễ thấy, do biến cố $A$ là một trường hợp nhỏ trong tổng thể tất cả trường hợp khi thực hiện phép thử, do đó $0 \leq k \leq n$. Nghĩa là \[0 \leq P(A) \leq 1\] với mọi biến cố $A$ bất kì.
\end{frame}

\begin{frame}{Ví dụ}
    \textbf{Ví dụ 1}. Xét phép thử tung 2 đồng xu. Gọi $A$ là biến cố 2 đồng xu cùng mặt.
    
    Ta ký hiệu $S$ là đồng xu sấp, $N$ là đồng xu ngửa. Khi đó các trường hợp có thể xảy ra của phép thử là $S-S$, $S-N$, $N-S$, $N-N$ (4 trường hợp). 
    
    Trong khi đó, các trường hợp có thể xảy ra của biến cố $A$ là $S-S$, $N-N$ (2 trường hợp).
    
    Kết luận: $P(A) = \frac{2}{4} = \frac{1}{2}$
    
    Chúng ta gọi tập hợp tất cả các trường hợp khi thực hiện phép thử là \textbf{không gian mẫu} và ký hiệu là $\Omega$. Mỗi phần tử trong không gian mẫu được gọi là \textbf{biến cố sơ cấp}. Trong bài này, $\Omega = \{S-S, S-N, N-S, N-N\}$.
    
    Tập hợp các trường hợp có thể xảy ra của biến cố gọi là \textbf{không gian biến cố} và ký hiệu là $\Omega_A$. Trong bài này $\Omega_A = \{S-S, N-N\}$.
    
    Như vậy, $P(A) = \frac{|\Omega_A|}{|\Omega|}$
\end{frame}

\begin{frame}{Ví dụ}
    \textbf{Ví dụ 2}. Tung 2 con súc sắc cân đối và đồng chất. Tính xác suất tổng số nút của 2 con súc sắc bằng 4.
    
    Việc tung mỗi con súc sắc có 6 trường hợp. Do đó $|\Omega| = 6^2 = 36$
    
    Gọi $A$ là biến cố tổng số nút của 2 con súc sắc bằng 4. Ta có các trường hợp là $4=1+3=3+1=2+2$ (3 trường hợp).
    
    Như vậy $|\Omega_A| = 3$ và $P(A) = \frac{3}{36} = \frac{1}{12}$
\end{frame}

\begin{frame}{Cộng và nhân xác suất}
    \textbf{Định nghĩa}: 2 biến cố được gọi là \textbf{xung khắc} nếu biến cố này xảy ra thì biến cố kia chắc chắn không xảy ra. Nói cách khác giao của chúng bằng rỗng.
    
    Khi đó, nếu $A$ và $B$ là 2 biến cố xung khắc, \[P(A + B) = P(A) + P(B)\]
    
    Ta còn có thể ký hiệu $P(A+B)$ là $P(A \cup B)$ (hợp 2 biến cố).
    
    \textbf{Định nghĩa}: 2 biến cố được gọi là \textbf{độc lập} nếu việc xảy ra của biến cố này không ảnh hưởng đến việc xảy ra của biến cố kia. 
    
    Khi đó, nếu $A$ và $B$ là 2 biến cố độc lập, \[P(AB) = P(A)P(B)\]
    
\end{frame}

\begin{frame}{Phép cộng xác suất mở rộng}
    Xét 2 tập hợp $A$ và $B$. Số phần tử của phép hợp 2 tập hợp trong trường hợp tổng quát được tính như sau: \[|A \cup B| = |A| + |B| - |A \cap B|\]
    
    Tương tự, xác suất của phép cộng xác suất đối với 2 biến cố có giao khác rỗng là: \[P(A + B) = P(A) + P(B) - P(A \cap B)\]
\end{frame}

\begin{frame}{Phép cộng xác suất mở rộng (Tổng quát)}
    Xét các tập hợp $A_1$, $A_2$, ..., $A_n$. Khi đó, số phần tử khi hợp các tập hợp này là:
    \begin{equation*}
        \begin{split}
            |A_1 \cup A_2 \cup \cdots \cup A_n| & = |A_1| + |A_2| + \cdots + |A_n| - \sum_{i, j}|A_i \cap A_j| \\ & + \sum_{i, j, k} |A_i \cap A_j \cap A_k| + \cdots \\ & = \sum_{i=1}^n (-1)^{i+1} \sum_{j_1, j_2, \cdots, j_i} |A_{j_1} \cap A_{j_2} \cap \cdots \cap A_{j_i} |
        \end{split}
    \end{equation*}
    
    Tương tự, đối với xác suất:
    \begin{equation*}
        P(A_1 \cup A_2 \cup \cdots \cup A_n) = \sum_{i=1}^n (-1)^{i+1} \sum_{j_1, j_2, \cdots, j_i} P(A_{j_1} \cap A_{j_2} \cap \cdots \cap A_{j_i})
    \end{equation*}
\end{frame}

\begin{frame}{Phép nhân xác suất mở rộng. Xác suất có điều kiện}
    Xét 2 biến cố $A$ và $B$. Khi đó xác suất xảy ra của biến cố $B$ với điều kiện biến cố $A$ xảy ra là: \[P(B | A) = \frac{P(AB)}{P(A)}\]
    Lúc này, $A$ và $B$ không độc lập.
    
    Tổng quát, nếu $n$ biến cố $A_i$, $i=1, \cdots, n$ không độc lập thì: \[P(A_1 A_2 \cdots A_n) = P(A_1)P(A_2|A_1)P(A_3 | A_2A_1)\cdots P(A_n|A_1A_2 \cdots A_{n-1})\]
\end{frame}

\begin{frame}{Ví dụ}
    \textbf{Ví dụ}. Xét 2 câu hỏi trắc nghiệm có 4 lựa chọn. Tính xác suất học sinh trả lời đúng câu thứ 2 với điều kiện câu đầu trả lời sai.
    
    \textbf{Giải}. Gọi $A$ là biến cố câu đầu tiên học sinh trả lời sai. $P(A) = \frac{3}{4}$
    
    Gọi $B$ là biến cố câu thứ 2 học sinh trả lời đúng. $P(B) = \frac{1}{4}$.
    
    Do $A$ và $B$ là 2 biến cố độc lập nên $P(AB) = P(A) P(B) = \frac{3}{16}$
    
    Như vậy, xác suất học sinh trả lời đúng câu 2 với điều kiện câu đầu trả lời đúng là: $P(B | A) = \frac{P(AB)}{P(A)} = \frac{3 / 16}{3 / 4} = \frac{1}{4}$
\end{frame}

\begin{frame}{Công thức xác suất đầy đủ}
    \textbf{Định nghĩa}: Xét phép thử có không gian mẫu là $\Omega$. Một hệ các biến cố $A_1$, $A_2$, ..., $A_n$ được gọi là \textbf{đầy đủ} nếu chúng thỏa các điều kiện:
    \begin{itemize}
        \item $A_1 \cup A_2 \cup \cdots \cup A_n = \Omega$
        \item $A_i \cap A_j = \emptyset$ với mọi $i \neq j$
    \end{itemize}
    
    Công thức xác suất đầy đủ: với biến cố $B$ bất kì trong phép thử: \[P(B) = P(A_1) P(B | A_1) + P(A_2) P(B | A_2) + \cdots P(A_n) P(B | A_n)\]
\end{frame}

\begin{frame}{Ví dụ về hệ biến cố đầy đủ}
    \textbf{Ví dụ}.
\end{frame}

\begin{frame}{Công thức Bayes}
    Xét hệ biến cố đầy đủ $\{A_1, A_2, \cdots, A_n\}$. Với biến cố $B$ bất kì thì: \[P(A_i | B) = \frac{P(A_i) P(B | A_i)}{\displaystyle{\sum_{j=1}^n P(A_j) P(B | A_j)}}\] với $1 \leq i \leq n$.
\end{frame}

\section{Xác suất liên tục}

\begin{frame}{Biến ngẫu nhiên}
    Xét phép thử với không gian mẫu $\Omega$. Với mỗi biến cố sơ cấp $\omega \in \Omega$ ta liên kết với 1 số thức $X(\omega)\in\mathbb{R}$ thì $X$ được gọi là \textbf{biến ngẫu nhiên}.
    
    \textbf{Định nghĩa}. \textbf{Biến ngẫu nhiên} $X$ của một phép thử với không gian mẫu $\Omega$ là ánh xạ: 
    \begin{equation*}
        \begin{split}
            X: & \Omega \mapsto \mathbb{R} \\ & \omega \mapsto X(\omega) = x
        \end{split}
    \end{equation*}
    
    Giá trị $x$ được gọi là một giá trị của biến ngẫu nhiên $X$.
    
    \begin{itemize}
        \item Nếu $X(\Omega)$ là 1 tập hữu hạn $\{x_1, x_2, \cdots, x_n\}$ hay vô hạn đếm được thì $X$ được gọi là \textbf{biến ngẫu nhiên rời rạc}.
        \item Nếu $X(\Omega)$ là 1 khoảng của $\mathbb{R}$ hay toàn bộ $\mathbb{R}$ thì $X$ được gọi là \textbf{biến ngẫu nhiên liên tục}.
    \end{itemize}
\end{frame}

\begin{frame}{Hàm mật độ của biến ngẫu nhiên rời rạc}
    Cho BNN rời rạc $X: \Omega \mapsto \mathbb{R}$, $X = \{x_1, x_2, \cdots, x_n, \cdots\}$. Giả sử $x_1 < x_2 < \cdots < x_n < \cdots$ với xác suất tương ứng là $P(\{\omega: X(\omega)=x_i\}) \equiv P(X=x_i) = p_i$, $i = 1, 2, \cdots$
    
    \textbf{Bảng phân phối xác suất của $X$}.
    
    \begin{table}
        \begin{tabular}{c|c c c c c}
            $X$ & $x_1$ & $x_2$ & $\cdots$ & $x_n$ & $\cdots$ \\ \hline $P$ & $p_1$ & $p_2$ & $\cdots$ & $p_n$ & $\cdots$
        \end{tabular}
    \end{table}
    
    \textbf{Hàm mật độ của $X$ là} \[f(x) = \begin{cases}
         p_i & \text{khi } x = x_i, \\
         0 & \text{khi } x \neq x_i, \forall i
    \end{cases}\]
    
    \textbf{Lưu ý}
    
    \begin{itemize}
        \item $p_i \geq 0$, $\sum p_i = 1$, $i = 1, 2, \cdots$

        \item $P(a < X \leq b) = \sum_{a < x_i \leq b}p_i$
    \end{itemize}
\end{frame}

\begin{frame}{Hàm mật độ của biến ngẫu nhiên liên tục}
    \textbf{Định nghĩa}. Hàm số $f: \mathbb{R} \mapsto \mathbb{R}$ được gọi là \textbf{hàm mật độ} của biến ngẫu nhiên liên tục $X$ nếu: \[P(a \leq X \leq b) = \displaystyle{\int_a^b f(x)\,dx}, \forall a, b \in \mathbb{R}\]
    
    \textbf{Nhận xét}. $\forall x \in \mathbb{R}$, $f(x) \geq 0$ và \begin{math}\int_{-\infty}^{+\infty}f(x)\,dx = 1\end{math}.
    
    \textbf{Ý nghĩa hình học}. Xác suất của biến ngẫu nhiên $X$ nhận giá trị trong $[a, b]$ bằng diện tích hình thang cong giới hạn bởi $x=a$, $x=b$, $y=f(x)$ và $Ox$.
\end{frame}


%------------------------------------------------

\begin{frame}{Bullet Points}
    \begin{itemize}
        \item Lorem ipsum dolor sit amet, consectetur adipiscing elit
        \item Aliquam blandit faucibus nisi, sit amet dapibus enim tempus eu
        \item Nulla commodo, erat quis gravida posuere, elit lacus lobortis est, quis porttitor odio mauris at libero
        \item Nam cursus est eget velit posuere pellentesque
        \item Vestibulum faucibus velit a augue condimentum quis convallis nulla gravida
    \end{itemize}
\end{frame}


%------------------------------------------------

\begin{frame}{Blocks of Highlighted Text}
    In this slide, some important text will be \alert{highlighted} because it's important. Please, don't abuse it.

    \begin{block}{Block}
        Sample text
    \end{block}

    \begin{alertblock}{Alertblock}
        Sample text in red box
    \end{alertblock}

    \begin{examples}
        Sample text in green box. The title of the block is ``Examples".
    \end{examples}
\end{frame}

%------------------------------------------------

\begin{frame}{Multiple Columns}
    \begin{columns}[c] % The "c" option specifies centered vertical alignment while the "t" option is used for top vertical alignment

        \column{.45\textwidth} % Left column and width
        \textbf{Heading}
        \begin{enumerate}
            \item Statement
            \item Explanation
            \item Example
        \end{enumerate}

        \column{.5\textwidth} % Right column and width
        Lorem ipsum dolor sit amet, consectetur adipiscing elit. Integer lectus nisl, ultricies in feugiat rutrum, porttitor sit amet augue. Aliquam ut tortor mauris. Sed volutpat ante purus, quis accumsan dolor.

    \end{columns}
\end{frame}

%------------------------------------------------
\section{Second Section}
%------------------------------------------------

\begin{frame}{Table}
    \begin{table}
        \begin{tabular}{l l l}
            \toprule
            \textbf{Treatments} & \textbf{Response 1} & \textbf{Response 2} \\
            \midrule
            Treatment 1         & 0.0003262           & 0.562               \\
            Treatment 2         & 0.0015681           & 0.910               \\
            Treatment 3         & 0.0009271           & 0.296               \\
            \bottomrule
        \end{tabular}
        \caption{Table caption}
    \end{table}
\end{frame}

%------------------------------------------------

\begin{frame}{Theorem}
    \begin{theorem}[Mass--energy equivalence]
        $E = mc^2$
    \end{theorem}
\end{frame}

%------------------------------------------------

\begin{frame}{Figure}
    Uncomment the code on this slide to include your own image from the same directory as the template .TeX file.
    %\begin{figure}
    %\includegraphics[width=0.8\linewidth]{test}
    %\end{figure}
\end{frame}

%------------------------------------------------

\begin{frame}[fragile] % Need to use the fragile option when verbatim is used in the slide
    \frametitle{Citation}
    An example of the \verb|\cite| command to cite within the presentation:\\~

    This statement requires citation \cite{p1}.
\end{frame}

%------------------------------------------------

\begin{frame}{References}
    % Beamer does not support BibTeX so references must be inserted manually as below
    \footnotesize{
        \begin{thebibliography}{99}
            \bibitem[Smith, 2012]{p1} John Smith (2012)
            \newblock Title of the publication
            \newblock \emph{Journal Name} 12(3), 45 -- 678.
        \end{thebibliography}
    }
\end{frame}

%------------------------------------------------

\begin{frame}
    \Huge{\centerline{\textbf{The End}}}
\end{frame}

%----------------------------------------------------------------------------------------

\end{document}