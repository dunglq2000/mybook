\section{Lý thuyết nhóm}

Câu chuyện bắt đầu vào một ngày khi mình vẫn còn sống ngày tháng tươi đẹp.

Cho tới khi học \textbf{lý thuyết nhóm} thì đời bớt đẹp hơn tí.

Để bắt đầu mình cần hiểu nhóm là gì.

\begin{defblock}{Nhóm (Group)}

Một tập hợp $G$ và toán tử 2 ngôi $\star$ trên $G$ tạo thành một nhóm nếu:
\begin{enumerate}
    \item Tồn tại phần tử $e \in G$ sao cho với mọi $g \in G$ thì $g \star e = e \star g = g$. Khi đó $e$ được gọi là \textbf{phần tử đơn vị} của $G$.
    \item Với mọi $g \in G$, tồn tại $g' \in G$ sao cho $g \star g' = g' \star g = e$. Khi đó $g'$ được gọi là \textbf{phần tử nghịch đảo} của $g$.
    \item Tính kết hợp: với mọi $a, b, c \in G$ thì $a \star (b \star c) = (a \star b) \star c$.
\end{enumerate}
\end{defblock}

\begin{defblock}{Nhóm Abel}
    
    Nếu nhóm $G$ có thêm tính giao hoán, tức là với mọi $a, b \in G$ thì $a \star b = b \star a$ thì $G$ gọi là nhóm giao hoán hay nhóm Abel
\end{defblock}

Lý thuyết nhóm thuộc toán trừu tượng, và nó trừu tượng thật. Tuy nhiên khi học về nó mình dần hiểu hơn về cách toán học vận hành và phát triển.

\newpage