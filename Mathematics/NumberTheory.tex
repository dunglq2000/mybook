\chapter{Mở đầu về số học}

Số học xuất hiện từ xa xưa, từ những bước đi đầu tiên của con
người. Tuy nhiên số học lại mang vẻ bí ẩn khó tưởng, sự phức
tạp vượt ra phạm vi số học. Nhà toán học vĩ đại Gauss từng nói
\textit{Toán học là vua của các môn khoa học, và số học là
nữ hoàng}. Hay một trong 23 bài toán thế kỷ của Hilbert về sự
phi mâu thuẫn của số học, người ta đã chứng minh được rằng
không thể chứng minh sự phi mâu thuẫn của số học chỉ bằng các
lý thuyết về số học.

\section{Phép chia Euclid}

Đây là nền tảng, cơ sở của số học. Từ khi biết tới phép chia
hai số nguyên, ta có thể tìm \textit{thương} và \textit{số dư}.
Nói theo toán học, nếu ta có 2 số nguyên dương $a$ và $b$, thì
tồn tại cặp số $q$, $r$ sao cho $a = qb + r$ với $0 \leq r < b$.

Khi đó, $a$ gọi là số bị chia, $b$ gọi là số chia, $q$ là thương
(q trong quotient) và $r$ là số dư (r trong remainder).

Đặc biệt là sự tồn tại của cặp số $q$ và $r$ là duy nhất. Thật
vậy, nếu ta giả sử tồn tại 2 cặp số $(q_1, r_1)$ và $(q_2, r_2)$ 
đều thỏa đẳng thức trên, nghĩa là
\[a = q_1 b + r_1, \quad a = q_2 b + r_2\]

Trừ 2 đẳng thức vế theo vế ta có $(q_1 - q_2) b + (r_1 - r_2) = 0$.
Tương đương $(r_2 - r_1) = (q_1 - q_2) b$, mà $0 \leq r_1, r_2 < b$
nên $-b < r_2 - r_1 < b$. Như vậy chỉ có thể xảy ra trường hợp
$r_2 - r_1 = 0$ hay $r_2 = r_1$, kéo theo $q_1 = q_2$.

\section{Thuật toán Euclid}

Dựa trên phép chia Euclid, ta có một thuật toán hiệu quả để tìm
ước chung lớn nhất giữa hai số $a$ và $b$.

Ký hiệu $\gcd(a, b)$ là ước chung lớn nhất của $a$ và $b$. Chúng 
ta thực hiện đệ quy như sau:
\[\gcd(a, b) = \begin{cases}
    a, \quad & \text{nếu}\,b = 0 \\
    \gcd(b, a \bmod b), \quad & \text{nếu}\,b \neq 0
\end{cases} 
    \]

Điểm quan trọng ở thuật toán Euclid là thuật toán chắc chắn sẽ dừng
sau một số hữu hạn bước, và kết quả sẽ là ước chung lớn nhất của 2
số $a$ và $b$.