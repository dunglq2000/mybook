\chapter{Lý thuyết nhóm}

Câu chuyện bắt đầu vào một ngày khi mình vẫn còn sống ngày tháng tươi đẹp.

Cho tới khi học \textbf{lý thuyết nhóm} thì đời bớt đẹp hơn tí.

Để bắt đầu mình cần hiểu nhóm là gì.

\section{Nhóm}

%\begin{defblock}{Nhóm (Group)}
\begin{definition}[Nhóm (Group)]
    Một tập hợp $G$ và toán tử 2 ngôi $\star$ trên $G$ tạo thành một nhóm nếu:
    \begin{enumerate}[noitemsep]
        \item Tồn tại phần tử $e \in G$ sao cho với mọi $g \in G$ thì $g \star e = e \star g = g$. Khi đó $e$ được gọi là \textbf{phần tử đơn vị} của $G$.
        \item Với mọi $g \in G$, tồn tại $g' \in G$ sao cho $g \star g' = g' \star g = e$. Khi đó $g'$ được gọi là \textbf{phần tử nghịch đảo} của $g$.
        \item Tính kết hợp: với mọi $a, b, c \in G$ thì $a \star (b \star c) = (a \star b) \star c$.
    \end{enumerate}
\end{definition}

%\end{defblock}

%\begin{defblock}{Nhóm Abel}
\begin{definition}[Nhóm Abel]
    Nếu nhóm $G$ có thêm tính giao hoán, tức là với mọi $a, b \in G$ thì $a \star b = b \star a$ thì $G$ gọi là nhóm giao hoán hay nhóm Abel
\end{definition}
%\end{defblock}

Lý thuyết nhóm thuộc toán trừu tượng, và nó trừu tượng thật. Tuy nhiên khi học về nó mình dần hiểu hơn về cách toán học vận hành và phát triển.

\begin{example}
    Xét tập hợp số nguyên $\ZZ$ và phép cộng 2 số nguyên.
    \begin{enumerate}[noitemsep]
        \item Phần tử đơn vị là 0 vì với mọi $a \in \ZZ$ thì $a + 0 = 0 + a = a$
        \item Với mọi $a \in \ZZ$, phần tử nghịch đảo là $-a$ vì $a + (-a) = (-a) + a = 0$
        \item Phép cộng số nguyên có tính kết hợp do đó thỏa mãn điều kiện về tính kết hợp
    \end{enumerate}
    Như vậy $(\ZZ, +)$ tạo thành nhóm. Lưu ý do phép cộng 2 số nguyên có tính giao hoán nên đây cũng là nhóm Abel.
\end{example}

\begin{example}
    Xét tập hợp số hữu tỉ khác 0 $\QQ^*$ và phép nhân 2 số hữu tỉ. Ta thấy do $a, b \in \QQ^*$ nên tích $a \cdot b$ cũng khác 0, do đó cũng thuộc $\QQ^*$.
    \begin{enumerate}[noitemsep]
        \item Phần tử đơn vị là 1 vì với mọi $a \in \QQ^*$ thì $a \cdot 1 = 1 \cdot a = a$
        \item Với mọi $a \in \QQ^*$, phần tử nghịch đảo là $\frac{1}{a}$ vì $a \cdot \frac{1}{a} = \frac{1}{a} \cdot a = 1$
        \item Phép nhân 2 số hữu tỉ có tính giao hoán do đó thỏa mãn điều kiện về tính kết hợp
    \end{enumerate}
    Tương tự như nhóm $\mathbb{Z, +}$, nhóm $(\QQ^*, \cdot)$ cũng là nhóm Abel.
\end{example}

\section{Nhóm con}

\begin{definition}[Nhóm con (Subgroup)]
    Cho nhóm $(G, \star)$. Tập hợp $H \subset G$ được gọi là \textit{nhóm con} của $G$ nếu với mọi $a, b \in H$ thì $a \star b \in H$
\end{definition}
 
Nghĩa là toán tử $\star$ đóng với các phần tử trong $H$.

\begin{example}
    Xét nhóm $(\ZZ, +)$ như trên. Ta xét tập con gồm các số chẵn của nó
    \[2\ZZ = \{\ldots, -4, -2, 0, 2, 4, \ldots\}\]

    Ta thấy rằng tổng 2 số chẵn vẫn là số chẵn, nghĩa là phép cộng số nguyên đóng trên $2\ZZ$.
    Do đó $(2\ZZ, +)$ là nhóm con của $(\ZZ, +)$.

    Như vậy mọi tập hợp có dạng $n \ZZ$ đều là nhóm con của $(\ZZ, +)$.
\end{example}

\section{Coset}

\begin{definition}[Coset]
    (tạm dịch - \textit{lớp kề} theo wikipedia) Cho nhóm $G$ và nhóm con $H$ của $G$.

    Coset trái của $H$ đối với phần tử $g \in G$ là tập hợp
    \[gH = \{gh : h \in H \}\]

    Tương tự, coset phải là tập hợp
    \[Hg = \{hg : h \in H \}\]
\end{definition}

Từ đây nếu không nói gì thêm ta ngầm hiểu là coset trái.

Ví dụ với nhóm con $2\ZZ$ của $\ZZ$, ta thấy rằng

\begin{enumerate}
    \item Nếu $g \in \ZZ$ là lẻ thì khi cộng với bất kì phần tử nào của $2\ZZ$ ta có số lẻ
    \item Nếu $g \in \ZZ$ là chẵn thì khi cộng với bất kì phần tử nào của $2\ZZ$ ta có số chẵn
\end{enumerate}

Nói cách khác, coset của $2\ZZ$ chia tập $\ZZ$ thành
\[0 (2\ZZ) = \{\ldots, -4, -2, 0, 2, 4, \ldots\}\]
 
\[1 (2\ZZ) = \{\ldots, -3, -1, 1, 3, \ldots \}\]

Trực quan mà nói, 2 coset trên rời nhau.

\begin{remark}
    Hai coset bất kì hoặc rời nhau, hoặc trùng nhau.
\end{remark}

\begin{proof}
    Nếu hai coset rời nhau thì không có gì phải nói. Ta chứng minh trường hợp còn lại.

    Giả sử $g_1 H \cap g_2 H \neq \emptyset$. Như vậy tồn tại $h_1, h_2 \in H$ mà $g_1 h_1 = g_2 h_2$.

    Do $h_1^{-1} \in H$, ta có $g_1 = g_2 h_2 h_1^{-1}$, nghĩa là $g_1 \in g_2 H$.

    Mà mọi phần tử trong $g_1 H$ có dạng $g_1 h$ nên $g_1 h = g_2 h_2 h_1^{-1} h$. Do $H$ là nhóm con của $G$ nên $h_2 h_1^{-1} h \in H$.
    Từ đó $g_1 H \subseteq g_2 H$. Tương tự ta cũng có $g_2 H \subseteq g_1 H$. Vậy $g_1 H = g_2 H$.
\end{proof}

\section{Normal Subgroup}

\begin{definition}[Normal Subgroup]
    (tạm dịch - \textit{nhóm con chuẩn tắc}) Nhóm con $H$ của $G$ được gọi là \textit{normal subgroup} nếu với mọi $g \in G$ ta có coset trái trùng với coset phải.
    \[gH = Hg \; \forall g \in G\]
\end{definition}

Nếu $H$ là normal subgroup của $G$ ta ký hiệu $H \triangleleft G$.

\begin{definition}[Quotient Group]
    (tạm dịch - \textit{nhóm thương}, hay Factor Group - \textit{nhóm nhân tử}). Với nhóm $G$ và normal subgroup của nó là $H$.
    Quotient Group được ký hiệu là $G / H$ và được định nghĩa là tập hợp các coset tương ứng với normal subgroup $H$.
    \[G / H = \{gH : g \in H \}\]

    Ta thấy rằng điều này chỉ xảy ra nếu $H$ là normal subgroup.
\end{definition}

\begin{example}
    Với nhóm $\ZZ$ và normal subgroup của nó là $2\ZZ$.
    Ta thấy $\ZZ / 2 \ZZ = \{0 + 2 \ZZ, 1 + 2 \ZZ\}$
\end{example}