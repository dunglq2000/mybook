\chapter{Lý thuyết vành}

\section{Vành}

\begin{definition}[Vành (Ring)]
    Cho tập hợp $R$, trên đó ta định nghĩa 2 toán tử \textit{cộng} và \textit{nhân}.

    Khi đó, $(R, +, \times)$ tạo thành vành nếu

    \begin{itemize}
        \item $(R, +)$ là nhóm Abel
        \item $(R, \times)$ có tính kết hợp với phép nhân. Với mọi $a, b, c \in R$ thì $a \times (b \times c) = (a \times b) \times c$
        \item Tính phân phối của phép cộng và phép nhân. Với mọi $a, b, c \in R$ thì $(a + b) \times c = a \times c + b \times c$
    \end{itemize}
\end{definition}

Tóm lại, $(R, +, \times)$ là vành nếu nó là nhóm Abel đối với phép cộng và có tính kết hợp với phép nhân.

\textbf{Lưu ý}. Phép nhân ở đây không nhất thiết có phần tử đơn vị, hay phần tử nghịch đảo như trong định nghĩa nhóm. Trong trường hợp này $(R, \times)$ gọi là semigroup (nửa nhóm).

\begin{definition}[Vành với đơn vị - Ring with identity] Nếu có phần tử $1_R \neq 0_R \in R$ sao cho với mọi $r \in R$ ta đều có
    \[1_R \times r = r \times 1_R = r\]
    thì $1_R$ được gọi là phần tử đơn vị đối với phép nhân.
\end{definition}

Ta thường ký hiệu $0_R$ là phần tử đơn vị của phép cộng $(R, +)$ và gọi là \textbf{phần tử trung hòa}.
Khi đó phần tử nghịch đảo của phép cộng gọi là \textbf{phần tử đối} và được ký hiệu là $-a$ nếu là đối của phần tử $a$.

Và $1_R$ là \textbf{phần tử đơn vị} đối với phép nhân $(R, \times)$.

\begin{definition}[Vành giao hoán - Commutative ring]
    Nếu ta có tính giao hoán đối với phép nhân, nghĩa là với mọi $a, b \in $ đều thỏa
    \[a \times b = b \times a\]
    thì ta nói là vành giao hoán (không cần nói rõ là phép nhân vì phép cộng bắt buộc phải giao hoán theo định nghĩa vành rồi).
\end{definition}

\section{Ideal}

\begin{definition}[Ideal]
    Xét vành $(R, +, \times)$. Một tập con $I$ của $R$ được gọi là 
    \textit{ideal trái} nếu
    \begin{itemize}
        \item $(I, +)$ là nhóm con của $(R, +)$
        \item với mọi $r \in R$, với mọi $x \in I$ thì $rx \in I$
    \end{itemize}
\end{definition}

Ta định nghĩa tương tự với ideal phải, khi đó $xr \in I$.
Từ đây về sau nếu không nói gì thêm nghĩa là mình xét ideal trái.

\begin{definition}[Ideal chính - Principal ideal]
    Nếu $I = aR$ với $a$ là phần tử nào đó trong $R$ thì $I$ được gọi 
    là  \textit{principal ideal}.
\end{definition}

Nói cách khác, nếu có một phần tử trong $R$ "sinh" ra được $I$ thì 
$I$ là principal.

\begin{definition}[Ideal cực đại - Maximal ideal]
    Nếu $I$ là một ideal của $R$ và không tồn tại tập con $I'$ mà 
    $I \subset I' \subset R$ (tập con thực thụ) thì $I$ được gọi
    là maximal ideal.
\end{definition}

\begin{corollary}
    Xét vành số nguyên $\ZZ$. Khi đó mọi ideal của $\ZZ$ đều là principal.
\end{corollary}

\begin{proof}
    Giả sử ideal $I$ của $\ZZ$ có phần tử dương nhỏ nhất là $n$.
    Theo định nghĩa của ideal thì với mọi $q \in \ZZ$ ta có 
    $qn \in I$.

    Nếu phần tử $a \in I$, theo phép chia Euclid ta có $a = qn + r$
    với $0 \leq r < n$, mà $a \in I$ và $qn \in I$ nên $r = a - qn \in I$.
    Tuy nhiên phần tử dương nhỏ nhất thuộc $I$ là $n$, do đó $r = 0$.

    Nói cách khác mọi phần tử $a \in I$ đều có dạng $qn$ với $q \in \ZZ$.

    Vậy mọi ideal đều là principal.
\end{proof}

\begin{corollary}
    Ideal $I$ của $\ZZ$ là maximal khi và chỉ khi $I = n\ZZ$ với
    $n$ là số nguyên tố.
\end{corollary}

\begin{proof}
    Ta chứng minh chiều thuận, chiều ngược tương tự. Sử dụng phản chứng,
    ta giả sử $n$ là hợp số. Khi đó $n = n_1 n_2$ ($n_1 \geq n_2 > 1$).

    Khi đó $n \ZZ \subset n_1 \ZZ \subset \ZZ$, suy ra ideal không phải
    maximal. Ta có điều phải chứng minh.
\end{proof}