\chapter{Không gian vector}

Phần này định nghĩa tổng quát của không gian vector.

\section{Không gian vector}

Xét tập hợp các vector $\mathcal{V}$ trên trường $\FF$.

Ta định nghĩa hai phép tính cộng và nhân trên các vector này sao cho

\begin{itemize}[noitemsep]
    \item Phép cộng là một ánh xạ $\mathcal{V} \times \mathcal{V} \to \mathcal{V}$ sao cho
    với mọi $\bm{x}, \bm{y} \in \mathcal{V}$ thì $\bm{x} + \bm{y} \in \mathcal{V}$
    \item Phân nhân vô hướng là ánh xạ $\FF \times \mathcal{V} \to \mathcal{V}$ sao cho
    với mọi $\alpha \in \FF$ và $\bm{x} \in \mathcal{V}$ thì $\alpha \bm{x} \in \mathcal{V}$
\end{itemize}

Nói cách khác, phép cộng 2 vector và phép nhân vô hướng 1 số với vector cho kết quả vẫn nằm trong không gian vector đó.

Đồng thời, phép cộng và phép nhân vô hướng phải thỏa mãn các tính chất sau

\begin{enumerate}[noitemsep]
    \item Tính giao hoán với phép cộng: với mọi $\bm{x}, \bm{y} \in \mathcal{V}$, $\bm{x} + \bm{y} = \bm{y} + \bm{x}$
    \item Tính kết hợp với phép cộng: với mọi $\bm{x}, \bm{y}, \bm{z} \in \mathcal{V}$, $\bm{x} + (\bm{y} + \bm{z}) = (\bm{x} + \bm{y}) + \bm{z}$
    \item Phần tử đơn vị của phép cộng: tồn tại vector không $\bm{0} \in \mathcal{V}$ sao cho với mọi $\bm{x} \in \mathcal{V}$, $\bm{0} + \bm{x} = \bm{x} + \bm{0} = \bm{x}$
    \item Phần tử đối của phép cộng: với mọi $\bm{x} \in \mathcal{V}$, tồn tại phần tử $\bm{x'} \in \mathcal{V}$ sao cho $\bm{x} + \bm{x'} = \bm{x} + \bm{x'} = \bm{0}$
    \item Phần tử đơn vị của phép nhân vô hướng: tồn tại phần tử $1_F \in \FF$ sao cho với mọi $\bm{x} \in \mathcal{V}$ thì $1_F \cdot \bm{x} = \bm{x}$
    \item Tính kết hợp của phép nhân vô hướng: với mọi $\alpha, \beta \in \FF$, với mọi $\bm{x} \in \mathcal{V}$ thì $\alpha (\beta \bm{x}) = (\alpha \beta) \bm{x}$
    \item Tính phân phối giữa phép cộng và nhân: với mọi $\alpha \in \FF$, với mọi $\bm{x}, \bm{y} \in \mathcal{V}$ thì $\alpha (\bm{x} + \bm{y}) = \alpha \bm{x} + \alpha \bm{y}$
    \item Tính phân phối giữa phép nhân vô hướng: với mọi $\alpha, \beta \in \FF$, với mọi $\bm{x} \in \mathcal{V}$ thì $(\alpha + \beta) \bm{x} = \alpha \bm{x} + \beta \bm{x}$
\end{enumerate}

Ta thấy rằng không gian vector xét ở chương trước xác định trên trường $\RR$.
Khi đó $\mathcal{V} = \RR^n$ và $\FF \equiv \RR$.

\section{Cơ sở và số chiều của không gian vector}

Tương tự, ta cũng có khái niệm cơ sở và số chiều của không gian vector.

Nếu trong không gian vector $\mathcal{V}$ tồn tại các vector độc lập tuyến tính $\bm{v_1}$, $\bm{v_2}$, ..., $\bm{v_d}$
mà tất cả các vector trong $\mathcal{V}$ có thể biểu diễn dưới dạng tổ hợp tuyến tính của các vector $\bm{v_i}$ trên,
thì tập hợp các vector 
\[\{ \bm{v}_1, \bm{v}_2, \ldots, \bm{v}_d \}\]
được gọi là \textbf{cơ sở} của không gian vector $\mathcal{V}$.

Khi đó,
\[\bm{x} = \sum_{i=1}^{d} \alpha_i \bm{v}_i \quad \forall \bm{x} \in \mathcal{V}\]

Số lượng phần tử của tập hợp các vector đó (ở đây là $d$) gọi là \textbf{số chiều (dimension)} của không gian vector $\mathcal{V}$.
Ta ký hiệu $\text{dim} \mathcal{V} = d$.

Ta còn ký hiệu 
\[\mathcal{V} = \text{span} \{\bm{v}_1, \bm{v}_2, \ldots, \bm{v}_d\}\]
và nói là không gian vector $\mathcal{V}$ được span (hay được sinh) bởi các vector $\bm{v_i}$.

Ta thấy rằng có thể có nhiều cơ sở cho cùng một không gian vector.

\begin{theorem}
    Mọi cơ sở của không gian vector $\mathcal{V}$ đều có số phần tử bằng $\text{dim} \mathcal{V}$
\end{theorem}

\section{Ví dụ về không gian vector}

Xét không gian vector $\RR$.

Ta biết rằng mọi điểm $M = (x, y)$ là tổ hợp tuyến tính 
của 2 vector $\vec{i} = (1, 0)$ và $\vec{j} = (0, 1)$.
Nghĩa là $(x, y) = x (1, 0) + y (0, 1)$. Như vậy $\vec{i}$
và $\vec{j}$ là cơ sở của $\RR^2$ và $\dim \RR^2 = 2$.

Cơ sở $\{\vec{i}, \vec{j}\}$ được gọi là \textbf{cơ sở chính tắc}
của $\RR^2$. Bây giờ xét 2 vector $(1, 2)$ và $(3, 4)$.

Ta thấy rằng $(1, 2) = 1 \cdot (1, 0) + 2 \cdot (0, 1)$
và $(3, 4) = 3 \cdot (1, 0) + 4 \cdot (0, 1)$. Viết dưới 
dạng ma trận là
\[
    \begin{pmatrix} 1 & 2 \\ 3 & 4 \end{pmatrix}
    \times \begin{pmatrix} 1 & 0 \\ 0 & 1 \end{pmatrix}
    = \begin{pmatrix} 1 & 2 \\ 3 & 4 \end{pmatrix}
\]
mà ma trận $\begin{pmatrix} 1 & 2 \\ 3 & 4 \end{pmatrix}$
khả nghịch. Vậy $(1, 2)$ và $(3, 4)$ cũng là một cơ sở của $\RR^2$.

\begin{remark}
    Ta thấy rằng trên $\RR^n$ luôn có hệ cơ sở chính tắc.
    Như vậy, hệ vector trên $\RR^n$ là cơ sở khi và chỉ khi 
    chúng độc lập tuyến tính, hay nói cách khác là khi viết 
    các hàng thành ma trận thì ma trận có định thức khác 0,
    hoặc rank ma trận bằng $n$.
\end{remark}

