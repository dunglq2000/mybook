\chapter{Lý thuyết trường}

\section{Trường}

\begin{definition}[Trường - Field]
    Cho tập hợp $F$ và hai toán tử 2 ngôi trên $F$ là phép cộng $+$ và phép nhân $\times$.
    Khi đó $(F, +, \times)$ là trường nếu
    \begin{itemize}[noitemsep]
        \item $(F, +, \times)$ là vành giao hoán với đơn vị
        \item Với mọi phần tử $f \neq 0_F$, tồn tại nghịch đảo $f^{-1}$ của $f$ đối với phép nhân. 
        Nghĩa là $f \times f^{-1} = f^{-1} \times f = 1_F$
    \end{itemize}
\end{definition}

Nói cách khác, $(F, \times)$ là nhóm Abel. Trên trường ta thực hiện được 4 phép tính cộng, trừ, nhân, chia.

\begin{example}
    Tập hợp số thực $\RR$ là một trường. Tập hợp các số phức $\CC$ là một trường.
    Tập hợp các số hữu tỷ dạng $a + b \sqrt{2}$ cũng là một trường.
\end{example}

Những trường trên được gọi là \textbf{trường vô hạn} vì có vô số phần tử.

Ngược lại, chúng ta cũng có các \textbf{trường hữu hạn}.

\begin{theorem}
    Gọi $R$ là vành giao hoán với đơn vị. Khi đó, nếu $I$ là ideal
    của $R$ thì $R / I$ là trường khi và chỉ khi $I$ là maximal ideal.
\end{theorem}

\begin{proof}
    Ta chứng minh điều kiện cần và điều kiện đủ.
    \begin{enumerate}
        \item Điều kiện cần. Ta có $I$ là maximal ideal. Ta thấy rằng
        $a + I \neq 0 \Leftrightarrow a \not\in I$. Vì nếu $a \in I$ thì 
        tồn tại $-a \in I$. Theo định nghĩa vành thì $a R$ cũng là ideal
        nên $I + a R$ là ideal, mà $a \not\in I$ và $a \in I + a R$ suy ra
        $I \subset I + a R$. Ta lại có $I$ là maximal nên $I + aR = R$, 
        do đó tồn tại $n \in I$ và $b \in R$ sao cho $n + ab = 1$. Tóm 
        lại là tồn tại nghịch đảo của phép nhân, do đó $R / I$ là trường.
        \item Điều kiện đủ. Với $R / I$ là trường. Ta giả sử $I$ không là
        maximal ideal. Khi đó tồn tại $I'$ sao cho $I \subset I' \subset R$.
        Khi đó tồn tại phần tử $a \in I'$ và $a \not\in I$ mà $a + I \neq 0$.
        Do đó $(a + I) (b + I) = 1 + I$ suy ra tồn tại $n \in I \subset I'$ 
        sao cho $a b = 1 + n$. Do $a, b \in I'$ nên $1 \in I'$, từ đó
        $1 \in R$ nên $I'$ không phải maximal. Ta có điều phải chứng minh.
    \end{enumerate}
\end{proof}

\section{Trường hữu hạn $GF(p)$}

Cho $p$ là số nguyên tố. Khi đó tập hợp các số dư khi chia cho $p$ 
cùng với phép cộng và nhân modulo $p$ tạo thành trường.

\begin{proof}
    Xét tập hợp các số dư khi chia cho $p$ là
    \[S = \{0, 1, \ldots, p-2, p-1\}\]
    Ta thấy rằng với mọi $a, b \in S$ thì $a + b \pmod p$ và $a \cdot b \pmod p$ đều thuộc $S$.

    \begin{itemize}[noitemsep]
        \item Vì $0 + a = a + 0 = a \pmod p$ với mọi $a \in S$ nên $0$ là phần tử đơn vị của phép cộng.
        \item Với mọi $a \in S$, ta có $(p-a) + a = a + (p-a) \equiv 0 \pmod p$ nên phần tử nghịch đảo của $a$ 
        đối với phép cộng là $p-a \in S$.
        \item Phép cộng modulo có tính kết hợp
        \item Phép cộng modulo có tính giao hoán
    \end{itemize}
    Như vậy $(S, +)$ là nhóm Abel.

    Tiếp theo, ta thấy rằng phép cộng và nhân có tính phân phối trên modulo.
    Đồng thời phép nhân modulo cũng có tính kết hợp. Do đó $(S, +, \cdot)$ là vành.

    \begin{itemize}[noitemsep]
        \item Phần tử đơn vị của phép nhân là 1
        \item Phép nhân modulo có tính giao hoán
        \item Do mọi phần tử thuộc $S$ đều nguyên tố cùng nhau với $p$ nên luôn tồn tại nghịch đảo của phần tử khác 0 trong $S$
    \end{itemize}
    Kết luận: $(S, +, \cdot)$ là trường.
\end{proof}

Ta thường ký hiệu trường này là $GF(p)$ (GF là viết tắt của Galois Field
để tưởng nhớ người có đóng góp quan trọng trong lý thuyết nhóm).

Trong mật mã học $GF(p)$ thường được sử dụng vì những tính toán 
của 2 quá trình ngược nhau là mã hóa và giải mã phải cho ra đúng
giá trị ban đầu. Nói cách khác việc tính toán không được vượt ra
ngoài một tập hợp nào đó. Việc lựa chọn trường $GF(p)$ cũng từ 
đó mà hiệu quả.

Tuy nhiên để đảm bảo an toàn thông tin, số nguyên tố $p$ được chọn khá lớn
(khoảng 256 bit). Việc này làm chậm tốc độ tính toán, cài đặt phức tạp (đa 
số các ngôn ngữ lập trình chỉ hỗ trợ kiểu dữ liệu cơ bản 2, 4, 8 byte, nên
cần định nghĩa kiểu dữ liệu mới để tính toán trên 32 byte).

Chúng ta có một ý tưởng ổn áp hơn ở phần sau.

\section{Trường hữu hạn $GF(p^n)$}

Xét các đa thức với hệ số nguyên
\[f(x) = a_n x^n + a_{n-1} x^{n-1} + \ldots + a_2 x^2 + a_1 x + a_0\]

Ta thấy rằng phép cộng và nhân 2 đa thức tạo thành một vành giao hoán với đơn vị.
Thêm nữa vành này có vô số phần tử. Ta cần một phương án để số phần tử là hữu hạn,
và đồng thời là trường.

Với $p$ là số nguyên tố và $n$ là số nguyên dương. Mình xét
các đa thức có bậc tối đa là $n-1$ với hệ số nằm trong tập hợp các số dư khi chia cho $p$.
Như vậy mình có $p^n$ đa thức như vậy.

\begin{example}
    Với $p=3$ và $n=2$. Khi đó các đa thức có thể có là
    $0, 1, 2, x, x+1, x+2, 2x, 2x+1, 2x+2$
\end{example}

Tương tự với việc modulo cho một số nguyên tố, ở đây mình 
xét phép cộng và nhân trên modulo một đa thức tối giản (irreducible 
polynomial) có bậc $n$ (vì khi modulo một đa thức bậc bất kì cho 
đa thức bậc $n$ ta có đa thức bậc nhỏ hơn $n$). Đồng thời hệ số của
đa thức từ phép cộng và nhân cũng được modulo $p$ (nằm trong $GF(p)$).

Với trường hợp $p=3$ và $n=2$ ở trên mình có thể chọn đa thức modulo 
là $m(x) = x^2 + 2x + 2$. Khi đó bảng phép nhân (phép cộng 
khá đơn giản nên mình không viết) 2 đa thức bậc 
nhỏ hơn 2 trong modulo $m(x)$ là

\begin{table}
    \centering
    \begin{subtable}[h]{\textwidth}
        \centering
        \begin{tabular}{|c|c|c|c|c|c|}
            \hline
            & 0 & 1 & 2 & $x$ & $x+1$ \\
            \hline
            0 & 0 & 0 & 0 & 0 & 0 \\
            \hline
            1 & 0 & 1 & 2 & $x$ & $x+1$ \\
            \hline
            2 & 0 & 2 & 1 & $2x$ & $2x+2$\\
            \hline
            $x$ & 0 & $2x$ & $x+1$ & $2x$ & $2x+1$ \\
            \hline
            $x+1$ & 0 & $x+1$ & $2x+2$ & $2x+1$ & 2\\
            \hline
            $x+2$ & 0 & $x+2$ & $2x+1$ & 1 & $x$ \\
            \hline
            $2x$ & 0 & $2x$ & $x$ & $2x+2$ & $x+2$ \\
            \hline
            $2x+1$ & 0 & $2x+1$ & $x+2$ & 2 & $2x$ \\
            \hline
            $2x+2$ & 0 & $2x+2$ & $x+1$ & $x+2$ & 1 \\
            \hline
        \end{tabular}
        \caption{Nửa đầu bảng nhân}
    \end{subtable}
    
    \begin{subtable}[h]{\textwidth}
        \centering
        \begin{tabular}{|c|c|c|c|c|}
            \hline
            & $x+2$ & $2x$ & $2x+1$ & $2x+2$ \\
            \hline
            0 & 0 & 0 & 0 & 0 \\
            \hline
            1 & $x+2$ & $2x$ & $2x+1$ & $2x+2$ \\
            \hline
            2 & $2x+1$ & $x$ & $x+2$ & $x+1$ \\
            \hline
            $x$ & 1 & $2x+2$ & 2 & $x+2$ \\
            \hline
            $x+1$ & $x$ & $x+2$ & $2x$ & 1 \\
            \hline
            $x+2$ & $2x+2$ & 2 & $x+1$ & $2x$ \\
            \hline
            $2x$ & 2 & $x+1$ & 1 & $2x+1$ \\
            \hline
            $2x+1$ & $x+1$ & 1 & $2x+2$ & $x$ \\
            \hline
            $2x+2$ & $2x$ & $2x+1$ & $x$ & 2 \\
            \hline
        \end{tabular}
        \caption{Nửa sau bảng nhân}
    \end{subtable}
    \caption{Bảng nhân trên $GF(3^2)$}
\end{table}

Ta thấy rằng bảng phép nhân đối xứng qua đường chéo chính. 
Điều này chứng minh phép nhân có tính giao hoán. 
Thêm nữa ở mỗi hàng hoặc cột khác 0 đều có 9 phần tử 
khác nhau.

