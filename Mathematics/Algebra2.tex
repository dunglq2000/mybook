\chapter{Một số loại hàm số}

Một số hàm số có tính chất đặc biệt giúp chúng ta tiết kiệm 
công sức trong chứng minh, tính toán.

\section{Hàm chẵn và hàm lẻ}

Xét hàm số $y=f(x)$ xác định trên miền $D$ có tính đối xứng, 
nghĩa là với mỗi phần tử dương $x$ thì có phần tử âm $-x$
hoặc ngược lại. Khi đó:

\begin{definition}
    Hàm số $y=f(x)$ được gọi là hàm số chẵn nếu với mọi $x \in D$
    ta có $f(-x) = f(x)$.
\end{definition}

Ví dụ như hàm số $y = x^2 + 3$ ở trên là một hàm chẵn vì với mọi
$x \in \RR$, $f(x) = x^2 + 3 = (-x)^2 + 3 = f(-x)$. Dễ thấy rằng
hàm chẵn đối xứng qua trục tung. Dựa vào tính chất này, 
trong lúc khảo sát hoặc tính toán đôi khi ta chỉ cần quan tâm 
một bên trục tung, bên kia tương tự (tính tích phân chẳng hạn).

\begin{definition}
    Hàm số $y=f(x)$ được gọi là hàm số lẻ nếu với mọi $x \in D$
    ta có $f(-x) = -f(x)$.
\end{definition}

Ví dụ như hàm số $y = \dfrac{1}{x}$ ở trên là một hàm lẻ vì với
mọi $x \in (0, +\infty)$, $f(-x) = \dfrac{1}{-x} = -\dfrac{1}{x} = 
-f(x)$. Dễ thấy rằng hàm lẻ đối xứng qua gốc tọa độ $O$.

\section{Hàm cộng tính}

Xét hàm số $y=f(x)$ xác định trên miền $D$. 

\begin{definition}
    Hàm số $y = f(x)$ được gọi là cộng tính nếu với mọi $x, y \in D$
    ta có $f(x+y) = f(x) + f(y)$.
\end{definition}

\begin{example}
    Hàm số $y = 2x$ trên $\RR$ là hàm cộng tính vì với mọi $x, y \in \RR$,
    ta có $f(x+y) = 2(x+y) = 2x + 2y = f(x) + f(y)$.
\end{example}

Hàm cộng tính có vai trò quan trọng trong giải tích, sẽ được trình bày
ở phần Giải tích phía sau.

\section{Hàm nhân tính}

Tương tự hàm cộng tính, ta định nghĩa hàm nhân tính.

\begin{definition}
    Hàm số $y = f(x)$ được gọi là hàm nhân tính nếu với mọi $x, y \in D$
    ta có $f(xy) = f(x) \cdot f(y)$.
\end{definition}

Hàm nhân tính quan trọng được sử dụng trong số học là phi-hàm Euler
về số lượng các số nguyên tố cùng nhau với số nguyên dương $n$. Hàm
Euler được trình bày ở phần Số học.

\section{Hàm tuần hoàn}

Xét hàm số $y=f(x)$ xác định trên miền $D$.

\begin{definition}
    Hàm số $y=f(x)$ được gọi là hàm tuần hoàn nếu tồn tại số $T$
    sao cho $f(x+T) = f(x)$ với mọi $x \in D$.
\end{definition}

Nói cách khác, hàm số sẽ lặp lại sau một đoạn nhất định.

Số $T$ nhỏ nhất thỏa mãn $f(x+T) = f(x)$ được gọi là \textbf{chu kỳ}
của hàm tuần hoàn. Vì sao lại là nhỏ nhất?

Ta thấy rằng, nếu $f(x+T) = f(x)$ với mọi $x \in D$, ta thay $x$ bởi
$x + T$ thì thu được $f(x + T + T) = f(x+T)$, hay $f(x+2T) = f(x+T)$.
Suy ra $f(x+2T) = f(x+T) = f(x)$. Như vậy thì sau $2T$ hàm số cũng
lặp lại đúng trạng thái đó. Tương tự cho $3T$, $4T$, .... Nên số $T$
nhỏ nhất thỏa mãn đẳng thức $f(x+T) = f(x)$ sẽ là chu kỳ.

\begin{example}
    Hàm số $y = \sin(x)$ là hàm tuần hoàn với chu kỳ $T = 2\pi$.
    Do đó chúng ta chỉ cần khảo sát hàm số trong khoảng $(-\pi, \pi)$
    thôi là đủ.
\end{example}