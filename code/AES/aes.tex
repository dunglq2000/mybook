\documentclass{article}

\usepackage{geometry}
\usepackage[utf8]{inputenc, vietnam}
\usepackage{amsmath, amsfonts}

\geometry{a5paper}

\title{Về Mix Column trong AES}
\author{Lê Quốc Dũng}

\begin{document}

Giả sử ma trận trạng thái trước khi bước vào phép tính Mix Column của AES
là 
\begin{equation}
    \begin{pmatrix}
        c_0 & c_1 & c_2 & c_3 \\
        c_4 & c_5 & c_6 & c_7 \\
        c_8 & c_9 & c_{10} & c_{11} \\
        c_{12} & c_{13} & c_{14} & c_{15}
    \end{pmatrix}
\end{equation}

Phép tính Mix Column lấy mỗi cột của ma trận trạng thái trên làm tham số cho đa thức
với hệ số trong $GF(2^8)$ và nhân với đa thức $c(z) = 2 + z + z^2 + 3z^3$ rồi modulo
cho $z^4 + 1$.

Giả sử với cột đầu tiên, ta viết hệ số theo thứ tự bậc tăng dần
$d(z) = c_0 + c_4 z + c_8 z^2 + c_{12} z^3$.

Tính (trong $GF(2^8)$)

\begin{align*}
    c(z) \cdot d(z) = & (2 + z + z^2 + 3 z^3) \cdot (c_0 + c_4 z + c_8 z^2 + c_{12} z^3) \\
    = & 2 c_0 + 2 c_4 z + 2 c_8 z^2 + 2 c_{12} z^3 \\
    + & c_0 z + c_4 z^2 + c_8 z^3 + c_{12} z^4 \\
    + & c_0 z^2 + c_4 z^3 + c_8 z^4 + c_{12} z^5 \\
    + & 3 c_0 z^3 + 3 c_4 z^4 + 3 c_8 z^5 + 3 c_{12} z^6 \\
    = & 2 c_0 + (2 c_4 + c_0) z + (2 c_8 + c_4 + c_0) z^2 \\
    + & (2 c_{12} + c_8 + c_4 + 3 c_0) z^3 + (c_{12} + c_8 + 3 c_4) z^4 \\
    + & (c_{12} + 3 c_8) z^5 + 3 c_{12} z^6
\end{align*}

Trong $GF(2^8)$ thì mọi phần tử đều có tính chất $2 x^n = 0$, tương đương với
$x^n = -x^n$. Do đó 
\begin{align*}
    & z^6 \pmod{z^4 + 1} \equiv -z^2 \equiv z^2 \\
    & z^5 \pmod{z^4 + 1} \equiv -z \equiv z \\
    & z^4 \pmod{z^4 + 1} \equiv -1 \equiv 1
\end{align*}

Suy ra
\begin{align*}
    c(z) \cdot d(z) = & 2 c_0 + (2 c_4 + c_0) z + (2 c_8 + c_4 + c_0) z^2 \\
    + & (2 c_{12} + c_8 + c_4 + 3 c_0) z^3 + (c_{12} + c_8 + 3 c_4) \\
    + & (c_{12} + 3 c_8) z + 3 c_{12} z^2 \\
    = & (c_{12} + c_8 + 3 c_4 + 2 c_0) + (c_{12} + 3 c_8 + 2 c_4 + c_0) z \\
    + & (3 c_{12} + 2 c_8 + c_4 + c_0) z^2 + (2 c_{12} + c_8 + c_4 + 3 c_0) z^3
\end{align*}

Như vậy xét hệ số lần lượt trước 1, $z$, $z^2$ và $z^3$ thì tương đương với phép
nhân ma trận

\begin{equation}
    \begin{pmatrix}
        2 & 3 & 1 & 1 \\
        1 & 2 & 3 & 1 \\
        1 & 1 & 2 & 3 \\
        3 & 1 & 1 & 2
    \end{pmatrix} \cdot
    \begin{pmatrix}
        c_0 \\ c_4 \\ c_8 \\ c_{12}
    \end{pmatrix}
\end{equation}
\end{document}