\chapter{Euclid}

Lúc mình học cấp 2, tiên đề Euclid được học là một trong 5 tiên đề hình học của Euclid. Nội dung tiên đề đó như sau:

\begin{axiom}[Tiên đề Euclid]
    Qua một điểm nằm ngoài đường thẳng cho trước, ta vẽ được một và chỉ một đường thẳng song song với đường thẳng đã cho.
\end{axiom}

Trong hình học Euclid, hình được vẽ trên \textit{mặt phẳng}. Ở đó, với 2 điểm phân biệt ta vẽ được duy nhất một đường thẳng đi qua 2 điểm đó.

Nếu chúng ta chỉ lấy phần ở giữa 2 điểm, ta có \textit{đoạn thẳng}. Nếu ta lấy phần ở ngoài 2 điểm nhưng chỉ một phía (đường thẳng kéo dài 2 phía) ta có nửa đường thẳng (hay còn gọi là tia).

Chúng ta có 2 công cụ để vẽ hình: thước và compa. Từ 2 công cụ này ta có thể vẽ được rất nhiều hình dạng như chia đôi góc (phân giác), chia đôi cạnh (lấy trung điểm), vẽ đường tròn, đường thẳng.

Tuy nhiên chúng lại có giới hạn: không thể chia 3 góc, hay không thể vẽ được hình đa giác đều 7 cạnh.

Những bài toán nhìn có vẻ đơn giản nhưng phải tới nhiều thế hệ sau, con người mới tìm được cách chứng minh rằng một hình nào đó có dựng được bằng thước và compa hay không.

Tiên đề là những mệnh đề mà ta thừa nhận tính đúng đắn của nó không cần chứng minh. Tuy nhiên sự đúng đắn phải được kiểm nghiệm từ thực tiễn.
Cơ sở của hình học Euclid gồm hệ các tiên đề làm nền móng cho các chứng minh toán học về sau.