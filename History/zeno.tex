\chapter{Zeno}

Zeno là nhà triết học nổi tiếng của Hy Lạp. Trong toán học, ông nổi tiếng về nghịch lý Zeno:

Archiles chạy đua với rùa. Do Archiles chạy nhanh hơn nên sẽ chấp rùa chạy trước. Khi đó Zeno bảo rằng Archiles sẽ không thể đuổi kịp rùa.

Phát biểu nghe rất mâu thuẫn nhưng được Zeno lý giải như sau:

\begin{itemize}[noitemsep]
    \item Giả sử ban đầu Archiles xuất phát sau con rùa một khoảng $d_1$
    \item Archiles mất một khoảng thời gian $t_1$ để đi hết quãng đường $d_1$ đó. Tuy nhiên trong khoảng thời gian $t_1$ đó con rùa cũng đi được một quãng đường $d_2$
    \item Archiles lại mất thêm một khoảng thời gian $t_2$ để đi hết quãng đường $d_2$. Nhưng rùa cũng đã đi được một đoạn $d_3$ nào đó trong thời gian $t_2$ rồi.
    \item Và cứ tiếp tục như thế, ta thấy rằng khoảng cách $d_n$ giữa 2 người sẽ nhỏ dần đi, nhưng không bao giờ chạm 0. Nói cách khác Archiles không thể bắt kịp con rùa.
\end{itemize}

Có gì đó rất \textit{không ổn} ở đây. Rõ ràng trên thực tế Archiles chắc chắn sẽ bắt kịp con rùa trong một khoảng thời gian nhất định. Nhưng tại sao suy luận của Zeno lại cho ra kết quả lạ thường vậy?

Câu trả lời là ở \textbf{vô cực}. Nói theo toán học hiện đại, khoảng cách $d_n$ tiến về 0 khi $n$ tiến ra vô cực.

Tuy nhiên sự vô cùng chưa được hiểu đúng ở thời của Zeno. Việc này sẽ được giải quyết ở thời của Cantor.