\chapter{Số học}

\section{Hàm Euler}

%\begin{defblock}{$\phi$ hàm Euler}
\begin{definition}[Phi hàm Euler]
    Cho số nguyên dương $n$. Số lượng các số dương nhỏ hơn $n$ và nguyên tố cùng nhau với $n$ được ký hiệu bởi $\phi(n)$ và gọi là $\phi$ hàm Euler. Nghĩa là
    \[\phi(n) = | \{ a : (a, n) = 1\} |\]
\end{definition}   
%\end{defblock}

Hàm Euler có ý nghĩa quan trọng trong lý thuyết số, công cụ giúp chúng ta giải các vấn đề về số mũ trong modulo.

Sau đây chúng ta xem xét hệ thặng dư đầy đủ và hệ thặng dư thu gọn.

Với số nguyên dương $n$, ta định nghĩa:

%\begin{defblock}{Hệ thặng dư đầy đủ}
\begin{definition}[Hệ thặng dư đầy đủ]
    Hệ thặng dư đầy đủ của $n$ là tập $\{0, 1, \ldots, n-1\}$.
\end{definition}
%\end{defblock}

Nói cách khác, hệ thặng dư đầy đủ của $n$ là các số dư có thể có khi chia một số bất kì cho $n$.

%\begin{defblock}{Hệ thặng dư thu gọn}

\begin{definition}[Hệ thặng dư thu gọn]
    Hệ thặng dư thu gọn của $n$ là tập các số $a$ mà $1 \leq a < n$ và $(a, n) = 1$. Số lượng các số $a$ như vậy là $\phi (n)$.  
\end{definition}

%\end{defblock}

\begin{remark}
    Hệ thặng dư thu gọn của $n$ gồm $\phi(n)$ phần tử là $$\{a_1, a_2, \ldots, a_{\phi(n)}\}$$
\end{remark}

\begin{remark}
    Nếu $n$ là số nguyên tố thì $\phi(n) = n-1$
\end{remark}

\section{Định lý Euler}

%\begin{theoremblock}{Định lý Euler}
    \begin{theorem}[Định lý Euler]    
        Cho số nguyên dương $n$. Với mọi số nguyên $a$ mà $(a, n) = 1$ thì 
        \[a^{\phi(n)} \equiv 1 \pmod n\]
    \end{theorem}
%\end{theoremblock}

\begin{proof}
    Giả sử $S = \{a_1, a_2, \ldots, a_{\phi(n)}\}$ là hệ thặng dư thu gọn của $n$. Ta sẽ chứng minh rằng nếu $a$ là số sao cho $(a, n)=1$ thì tập hợp
    $$\{a a_1, a a_2, \ldots, a a_{\phi(n)}\}$$ là hoán vị của tập $S$.

    Thật vậy, giả sử $a a_i \equiv a a_j \pmod n$ với $1 \leq i, j \leq \phi(n)$ và $i \neq j$.

    Do $(a, n) = 1$ nên tồn tại nghịch đảo $a' \pmod n$, nhân $a'$ cho 2 vế ta còn $a_i \equiv a_j \pmod n$.

    Nói cách khác, nếu $a_i \not\equiv a_j \pmod n$ thì $a a_i \not\equiv a a_j \pmod n$. Suy ra tập
    $$\{a a_1, a a_2, \ldots, a a_{\phi(n)}\}$$ là hoán vị của $S$.

    Ta nhân tất cả phần tử của $S$ thì sẽ bằng tích phần tử của tập trên
    $$a a_1 \cdot a a_2 \ldots a a_{\phi(n)} \equiv a_1 \cdot a_2 \ldots a_{\phi(n)} \pmod n$$

    Đặt $I = a_1 \cdot a_2 \ldots a_{\phi(n)}$ thì phương trình trên tương đương với 
    $$a^{\phi(n)} I \equiv I \pmod n$$
    
    Mà $(I, n) = 1$ do là tích các số nguyên tố cùng nhau với $n$ nên rút gọn 2 vế ta được
    $$a^{\phi(n)} \equiv 1 \pmod n$$

    Ta có điều phải chứng minh.
\end{proof}

\section{Định lý Fermat nhỏ}

%\begin{theoremblock}{Định lý Fermat nhỏ}
    \begin{theorem}[Định lý Fermat nhỏ]    
    Cho số nguyên tố $p$. Với mọi số nguyên $a$ thì $$a^p \equiv a \pmod p$$

    Khi $(a, p) = 1$ thì \[a^{p-1} \equiv 1 \pmod p\]
    \end{theorem}
%\end{theoremblock}

\begin{remark}
    Khi $(a, p) = 1$ thì định lý Fermat là hệ quả trực tiếp từ định lý Euler.
\end{remark}

\section{Tính chất hàm Euler}

\begin{remark}
    Với $(m, n) = 1$ thì $$\phi(m n) = \phi(m) \phi(n)$$
\end{remark}

\begin{proof}
    Ta viết các số từ 1 tới $mn$ thành bảng như sau

    \begin{center}
        \begin{tabular}{c c c c}
            1 & $m+1$ & $\cdots$ & $(n-1)m + 1$ \\
            2 & $m+2$ & $\cdots$ & $(n-1)m + 2$ \\
            $\cdots$ & $\cdots$ & $\cdots$ & $\cdots$ \\
            $m$ & $m+m$ & $\cdots$ & $(n-1)m + m$
        \end{tabular}
    \end{center}
    
    Hàng $r$ gồm các phần tử dạng $r m + k$ với $0 \leq r \leq n-1$ và $1 \leq k \leq m$. 
    Ta thấy rằng nếu $(rm + k, m) = 1$ thì $(k, m) = 1$.

    Do đó trên mỗi hàng có $\phi(m)$ phần tử nguyên tố cùng nhau với $m$.

    Tiếp theo, trên các hàng vừa tìm được, do $(m, n) = 1$ nên để $(rm + k, n) = 1$ thì $(r, n) = 1$.
    Nghĩa là có $\phi(n)$ hàng như vậy.

    Tổng kết lại, ta có $\phi(m) \phi(n)$ phần tử trong bảng nguyên tố cùng nhau với $mn$. Do đó có điều phải chứng minh.
\end{proof}