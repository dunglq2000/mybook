\chapter{Phép biến hình}

Trong thực tế chúng ta hay gặp các vấn đề về việc di dời một hình
nào đó sang một vị trí khác trong mặt phẳng, không gian và phải đảm bảo 
giữ nguyên một số quan hệ nhất định. Trong đó cơ bản nhất và được ứng dụng 
rộng rãi là phép dời hình và phép đồng dạng.

\section{Phép dời hình}

\begin{definition}[Phép dời hình]
    Phép dời hình từ hình $\mathcal{H}$ thành hình $\mathcal{H}'$ 
    là một ánh xạ $f$ biến mỗi điểm thuộc hình $\mathcal{H}$ thành điểm
    thuộc hình $\mathcal{H}'$ sao cho khoảng cách giữa 2 điểm bất
    kì trong $\mathcal{H}$ bảo toàn khi qua $\mathcal{H}'$.
\end{definition}

Nói cách khác, với mọi điểm $A, B \in \mathcal{H}$, ánh xạ $f$
biến $A$ thành $A'$ và $B$ thành $B'$ ($A', B' \in \mathcal{H}'$)
thì $A'B' = AB$.

Chúng ta thường thấy việc dời hình theo vector (dời theo 1 hướng nhất định),
đối xứng qua trục, đối xứng qua tâm, quay quanh tâm hoặc trục nào đó.

\section{Phép dời hình theo vector}

Phép dời hình theo vector $\vec{v} \neq \vec{0}$ biến
điểm $A$ thành điểm $A'$ sao cho $\overrightarrow{AA'} = \vec{v}$.

Dễ thấy đây là phép dời hình vì với mọi $A, B$ biến thành $A', B'$ ta có
$\overrightarrow{A'B'} = \overrightarrow{A'A} + \overrightarrow{AB} + \overrightarrow{BB'}$
mà ta có $\overrightarrow{A'A} = -\vec{v} = -\overrightarrow{BB'}$ nên
$\overrightarrow{A'B'} = \overrightarrow{AB}$. Vector bằng nhau thì độ dài cũng bằng nhau.
Ta có điều phải chứng minh.

\section{Phép đối xứng qua tâm cố định}

Cho điểm cố định $O$. Phép đối xứng tâm $O$ biến điểm $A$ thành điểm $A'$ sao
cho $\overrightarrow{OA} = -\overrightarrow{OA'}$. Nói cách khác $O$ là trung điểm
đoạn thẳng $AA'$.

\section{Phép quay quanh tâm cố định}

Cho điểm cố định $O$. Phép quay (mặc định là ngược chiều đồng hồ) quanh tâm $O$
theo một góc cố định $\varphi$ biến điểm $A$ thành điểm $A'$ sao cho
$\widehat{(\overrightarrow{OA}, \overrightarrow{OA'})} = \varphi$.

Trên mặt phẳng chúng ta có thể biểu diễn phép quay dưới hệ tọa độ như sau.

Giả sử vector $\overrightarrow{OA}$ có độ dài là $r$ và hợp với trục $Ox$ một góc $\alpha$.

Khi đó, giả sử tọa độ của $\overrightarrow{OA} = (x, y)$ thì ta có 
\begin{align*}
    x & = r \cos \alpha \\
    y & = r \sin \alpha
\end{align*}

Nếu ta quay vector này quanh gốc tọa độ, ngược chiều kim đồng hồ một góc $\varphi$ thì 
thực ra góc (mới) hợp bởi vector $\overrightarrow{OA'}$ và trục $Ox$ là $\alpha + \varphi$.
Do đó
\begin{align*}
    x' & = r \cos (\alpha + \varphi) \\
    y' & = r \sin (\alpha + \varphi)
\end{align*}

Khi khai triển ra,
\begin{align*}
    x' & = r \cos \alpha \cos \varphi - r \sin \alpha \sin \varphi
= x \cos \varphi - y \sin \varphi \\
    y' & = r \sin \alpha \cos \varphi + r \cos \alpha \sin \varphi 
= y \cos \varphi + x \sin \varphi
\end{align*}

Như vậy viết dưới dạng ma trận
\[\begin{pmatrix}
    x' \\ y'
\end{pmatrix} = \begin{pmatrix}
    \cos \varphi & -\sin \varphi \\
    \sin \varphi & \cos \varphi
\end{pmatrix} \begin{pmatrix}
    x \\ y
\end{pmatrix}\]

Dễ thấy, phép quay bảo toàn khoảng cách từ tâm $O$ tới điểm đó. Nghĩa là $OA = OA'$.