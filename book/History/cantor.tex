\chapter{Georg Cantor}

\begin{center}
    \textit{Nhà toán học vĩ đại bị vùi lấp trong sự bảo thủ} - L.Q. Dũng
\end{center}

\section{Dẫn nhập}

Con người luôn tìm tòi, học hỏi, sáng tạo để tiến lên.
Tuy nhiên luôn tồn tại những định chế bảo thủ, cố chấp theo
lối mòn kìm hãm sự phát triển của những thiên tài, những ý 
tưởng cách mạng. Cantor có lẽ đã sinh sai thời, hoặc những phát kiến của ông
quá vượt trội so với thời đại, khiến những định chế cấp cao
khó lòng chấp nhận.

Georg Cantor (1845-1918), người được thầy Lê Quang Ánh gọi 
là \textit{người chế ngự vô cực}, là nhà toán học vĩ đại mà
mình rất nể phục.

Khi học về tổ hợp, các phương pháp đếm ở phổ thông, thầy giáo bảo lớp mình
"đếm" thử xem, số lượng phần tử của tập hợp $\NN$ và $\ZZ$ cái
nào nhiều hơn. Ban đầu ai cũng nghĩ rằng $\ZZ$ lớn hơn vì rõ ràng
$\NN$ là tập con của $\ZZ$. Tuy nhiên thầy đã giải thích như sau:

Chúng ta "móc nối" giữa một phần tử thuộc $\NN$ và một phần tử thuộc $\ZZ$
theo quy tắc
\begin{itemize}[noitemsep]
    \item Ta ghép một số chẵn của $\NN$ với một và chỉ một số dương của $\ZZ$.
    Ví dụ $0 \to 0$, $2 \to 1$, $4 \to 2$, và tương tự;
    \item Ta ghép một số lẻ của $\NN$ với một và chỉ một số âm của $\ZZ$.
    Ví dụ $1 \to -1$, $3 \to -2$, $5 \to -3$, và tương tự.
\end{itemize}

Thật kỳ lạ, nếu theo quy tắc này, với mọi số thuộc $\NN$ ta luôn tìm được 
duy nhất một người bạn bên $\ZZ$. Ngược lại với mỗi số thuộc $\ZZ$ ta cũng
tìm ngược lại được một người đồng hành bên $\NN$. Nói theo toán học thì 
đây là một song ánh. Như vậy hai tập hợp có số phần tử (lực lượng) bằng nhau.

Về sau mình biết được người nghĩ ra phương pháp này là Cantor.
Xét theo thời đó, đây là một phát kiến bất ngờ gây tiếng vang 
lớn thời đó.

Chúng ta học đếm từ nhỏ: đếm 1, 2, 3 viên bi; đếm 1, 2, 3 quả táo; vân vân.
Khi một tập hợp có hữu hạn phần tử, ta đếm được số lượng phần tử của tập hợp đó.
Nhưng nếu là một tập vô hạn thì sao? Chúng có đếm được không? Mà đếm là sao?

Chúng ta quay lại nguồn gốc toán học. Các con số đã được sử dụng từ xa xưa,
nhất là các số tự nhiên bởi vì chúng ... tự nhiên xuất hiện. Chúng ta đếm
số lượng viên bi, số lượng quả táo thông qua các số tự nhiên, mà 
tập hợp $\NN$ là vô hạn. Chúng ta cứ đếm lên 1, rồi lại đếm lên 1, 
cứ như vậy mãi mãi. Như vậy có thể thấy tập $\NN$ là tập vô hạn đếm được.

Cantor đã chỉ ra rằng, một tập hợp vô hạn gọi là đếm được nếu
tồn tại một song ánh từ $\NN$ tới nó. Ngược lại thì là tập không đếm được.
Quay lại vấn đề về tập $\ZZ$ và $\NN$
ban nãy thì song ánh từ $\ZZ$ tới $\NN$ (nếu $\phi$ là song ánh
thì ánh xạ ngược $\phi^{-1}$ cũng là song ánh) là:

\begin{equation*}
    f(z) = \begin{cases}
        2z, & z \geq 0 \\
        -1-2z, & z < 0
    \end{cases}    
\end{equation*}

Bằng lý luận tương tự, tập hợp số hữu tỷ $\QQ$ cũng là tập
đếm được do có song ánh từ $\ZZ \times \NN$ tới $\QQ$ (tử số
và mẫu số).

\section{Cantor tiến xa}

Bây giờ chúng ta xem xét tập $\RR$, nơi hội tụ của vô số anh
hào hữu tỷ, vô tỷ, siêu việt, v.v. Chúng ta không thể kiểm soát
các số vô tỷ, đơn giản vì chúng không có cấu trúc đặc biệt gì cho chúng.
Chẳng hạn như $\ZZ$ rời rạc kéo dài về 2 phía, hoặc $\QQ$
là tích Descartes của $\ZZ$ và $\NN$. Cantor sẽ giúp chúng ta 
gỡ bỏ những rắc rối này.

Đầu tiên, Cantor chứng minh rằng tập $\RR$ là tương đương khoảng
$(0, 1)$. Chúng ta dễ thấy hàm số $f: \RR \to (0, 1)$ cho
bởi công thức $f(x) = \frac{e^x}{e^x + 1}$ là song ánh.
Như vậy công việc cần làm là chứng minh không tồn tại song
ánh từ khoảng $(0, 1)$ tới $\NN$ nữa là xong. Khi đó $(0, 1)$
sẽ là tập không đếm được.

Cantor đưa ra hai phép chứng minh, cả hai đều chưa từng có và
gây ra tiếng vang lớn thời đó.

\begin{proof}[Phương pháp đường chéo]
    Giả sử $(0, 1)$ là tập đếm được, nhưg vậy tồn tại song ánh từ $\NN$ tới $(0, 1)$.
    Cantor chứng minh toàn ánh không xảy ra.

    Với mọi số tự nhiên $n$, ta xét các số thực khác nhau thuộc $(0, 1)$

    \begin{align*}
        1 & \to a_1 = 0.a_{11} a_{12} \ldots a_{1n} \ldots \\
        2 & \to a_2 = 0.a_{21} a_{22} \ldots a_{2n} \ldots \\
        & \ldots \\
        i & \to a_i = 0.a_{i1} a_{i2} \ldots a_{in} \ldots \\
        & \ldots \\
        n & \to a_n = 0.a_{n1} a_{n2} \ldots a_{nn} \ldots
    \end{align*}

    Bây giờ ta chọn số $b = 0.b_1 b_2 \ldots b_n \ldots$ sao cho
    $b_i \neq a_{ii}$. Nghĩa là $b$ khác $a_i$ ở vị trí thứ $i$,
    từ đó $b \neq a_i$ với mọi $a_i$.
    Nhưng như vậy thì không có $n$ nào là biến thành $b$. Vậy không tồn
    tại toàn ánh và như vậy ta có điều phải chứng minh.
\end{proof}

\begin{proof}[Phương pháp dãy các khoảng kín bị chặn lồng vào nhau]
    Như trên, ta vẫn giả sử tập $(0, 1)$ đếm được. Khi đó tồn tại song ánh
    từ $(0, 1)$ tới $\NN$, và do vậy ta có thể liệt kê (một cách vô hạn)
    các phần tử của $(0, 1)$ như việc viết các số tự nhiên:
    \[I = (0, 1) = \{x_1, x_2, x_3, \ldots \}\]

    Đầu tiên ta chọn một khoảng kín $I_1$ trong $I$ sao cho $x_1 \not\in I_1$.
    Sau đó ta chọn một khoảng kín $I_2$ trong $I_1$ sao cho $x_2 \not\in I_2$.
    Cứ tiếp tục như vậy ta được dãy các khoảng kín lồng vào nhau
    \[ \ldots I_n \subseteq I_{n-1} \subseteq \ldots \subseteq I_2 \subseteq I_1 \subseteq I\]
    sao cho $x_n \not\in I_n$ với mọi số tự nhiên $n$. Theo tính chất của dãy 
    các khoảng kín bị chặn lồng vào nhau thì giao của tất cả $I_k$ ($k = 1, 2, \ldots $)
    khác rỗng. Hay nói cách khác tồn tại $\alpha \in I_n$ với mọi số tự nhiên
    $n$. Khi đó $\alpha \neq x_i$ với $i=1, 2, \ldots, n$, nghĩa là $\alpha$
    khác với mọi phần tử được liệt kê ở trên. Điều này vô lý vì $\alpha \in I_n$ nên
    $\alpha \in I$. Như vậy tập $(0, 1)$ là không đếm được.
\end{proof}

Từ chứng minh trên ta có định lý Cantor:

\begin{theorem}
    Tập hợp các số vô tỷ không đếm được.
\end{theorem}

\section{Thiên tài và bi kịch}

Như đầu bài đã nói, những phát kiến vượt quá thời đại thường bị những
phe phái bảo thủ chống đối và kìm hãm phát triển. Người thầy cũ, cũng 
là gây nên vết thương đau đớn nhất cho Cantor là nhà toán học 
Leopold Kronecker (1823 - 1891).

Leopold Kronecker thường được biết đến với ký hiệu Leopold Kronecker 
trong thặng dư chính phương (như Legendre hay Jacobi). Ông là nhà toán
học người Đức nổi tiếng là bảo thủ. Ông thậm chí còn cho rằng \textit{
    "Thượng Đế làm ra số nguyên, tất cả còn lại là do con người."
}. Theo cách nói của ông, những gì không được xây dựng trên cơ sở
các số nguyên đều là lố bịch, vớ vẩn. Và giải tích tất nhiên là nạn 
nhân của định kiến này. Giải tích do Wierstrass tạo ra như cái gai trong
mắt ông vì giải tích liên quan đến sự liên tục, tới những thứ vô cùng nhỏ
như trong ngôn ngữ $\delta - \varepsilon$ của Cauchy, điều mà Kronecker 
cũng thấy không ưa. Nhưng mà, thưa Kronecker tài ba, từ thời Pythargoras
đã tìm ra số vô tỉ $\sqrt{2}$. Như vậy nếu ông từ chối lý thuyết tập hợp 
của Cantor, thì có phải ông vừa quăng sự phát triển toán học hàng thế kỷ 
về với gốc rễ của nó? Sự thật là đúng như vậy. Nếu phủ nhận lý thuyết tập
hợp của Cantor cũng chính là phủ nhận sự tồn tại của các số vô tỉ
(theo lời Cantor).

Nhưng thế lực của Kronecker lúc đó quá mạnh. Còn Cantor chỉ là giảng
viên ở một trường đại học hạng hai, luôn muốn có nhiệm sở tại trung 
tâm khoa học của Đức - Đại học Berlin. Kronecker thậm chí còn dùng
quyền lực gây ảnh hưởng lên các tòa soạn khiến các bài báo của Cantor
không được đăng ở các tạp chí uy tín mà chỉ có thể đăng ở các tạp chí 
hạng thấp. Cuộc sống chật vật khó khăn, cộng thêm áp lực trong thời gian
dài không thể chứng minh \textit{giả thiết về sự liên tục} đã làm Cantor
kiệt sức.

Hơn thế nữa, ông trời có vẻ rất thích đùa giỡn với những thiên tài.
Năm 1899, người con trai út của ông - người con ông yêu thương nhất - 
đột ngột qua đời. Chấn động đó làm ông đột quỵ. Sau đó ông cứ nhập viện,
xuất viện nhiều lần và cuối cùng rời nhiệm sở trong tình trạng gần như 
mất trí.

Tuy nhiên, chân lý của ông, lý thuyết của ông lúc này đã được đón nhận
khắp mọi nơi sau chiến thắng của phe David Hilbert, người được gọi là
nhà thông thái cuối cùng của thế kỷ 20. Lý thuyết của Cantor ban đầu đã
được các nhà toán học tài năng thời đó ủng hộ như Dedekind, Wierstrass, Hilbert, 
... . Tuy nhiên phe bảo thủ của Kronecker quá mạnh nên sự giúp đỡ của
họ cho Cantor không đủ để thắng phe bảo thủ. Và rồi cái gì cần tới cũng phải
tới. Những suy luận đúng đắn, những lập luận chặt chẽ sẽ mang tới kết quả
đúng đắn, dù nó có khó tin tới đâu đi nữa. Các nhà toán học cuối cùng cũng 
đi tới kết luận rằng lý thuyết tập hợp của Cantor là mũi tên vững chắc
mở đường cho toán học phát triển.

Cantor đã ra đi mãi mãi tại bệnh viện vào ngày 6 tháng 1 năm 1918, để
lại cho đời sau nhiều ý tưởng đột phá. Hilbert đã từng nói về lý thuyết 
của Cantor rằng \textit{"Đó là một sản phẩm trí tuệ tinh tế nhất của một 
thiên tài Toán học, một trong thành tựu cao cấp nhất mà trí tuệ con người
có thể đạt được."} (Burton). Hilbert cũng nói thêm rằng \textit{
    "Không ai có thể ngăn cấm chúng ta bước vào thế giới kỳ diệu
    mà Cantor đã tạo ra."
}(Dunham).