Trong lịch sử, từ xa xưa con người đã biết tính toán, sử dụng chúng cho công việc hằng ngày.

Chúng ta không biết ai là người đầu tiên phát minh ra lịch, cũng như cách tính toán để phân chia ruộng đất, tài sản trong các nền văn minh cổ.
Những điều đó được đúc kết theo kinh nghiệm qua hàng chục, thậm chí hàng trăm năm tri thức con người.

Cho tới khi những nhân vật sau (và nhiều nhân vật tương tự khác) đi du lịch Ai Cập và phương đông (ý mình là đi du học).

Đầu tiên phải nhắc tới Euclid, người đã quá quen thuộc với học sinh phổ thông với tiên đề Euclid. Hệ tiên đề Euclid đề ra trở thành cơ sở cho hình học. Bộ sách \textit{Elements} của ông được cho là bộ sách giáo khoa đầu tiên trên thế giới và những gì ghi trong đó khá giống với những gì được giảng dạy ở trường học chúng ta ngày nay.

Nhưng ông đã không lường trước được 1 điều: thế hệ sau đã "thêm mắm dặm muối" và biến đổi hình học của ông thành hình học Phi-Euclid. Từ đó mở ra những khả năng lớn hơn của toán học.

Pythagoras: định lý Pythagoras trong tam giác vuông có lẽ là định lý đầu tiên mà học sinh tiếp cận. Phát biểu rất đơn giản:

\begin{theorem}[Định lý Pythagoras]
    Trong tam giác vuông, bình phương cạnh huyền bằng tổng bình phương hai cạnh góc vuông.

    Nói cách khác, tam giác có 2 cạnh góc vuông lần lượt là $a$ và $b$, cạnh huyền độ dài là $c$ thì
    \[a^2 + b^2 = c^2\]
\end{theorem}

Thật ra trước thời Pythagoras rất lâu, người Ai Cập đã biết tới phương pháp này. Có nhiều bằng chứng về các cuộn giấy papyrus ghi lại các bộ số nguyên $(a, b, c)$ mà $a^2 + b^2 = c^2$ được tìm thấy khi khai quật.

Tuy nhiên thời đó con người chỉ làm việc với các số nguyên, chính xác hơn là các số tự nhiên vì chúng "tự nhiên" xuất hiện trong đời sống.

Pythagoras là người đầu tiên nhắc tới \textbf{proof} (chứng minh) trong toán học. Một phát biểu, định lý chỉ đúng khi có một chứng minh đúng đắn cho nó. Các bước suy luận trong chứng minh dựa trên một hệ tiên đề (axiom) cho trước.
Các tiên đề này hiển nhiên đúng, từ đó các suy luận chính xác sẽ cho kết quả chính xác.

Cho tới khi Fermat phán:

\begin{theorem}[Định lý cuối cùng của Fermat]
    Không tồn tại một cách phân tích tam thừa thành tổng 2 tam thừa, tứ thừa thành tổng 2 tứ thừa, hay tổng quát hơn

    Với mọi số nguyên $n \geq 3$, không tồn tại bộ số nguyên $(a, b, c)$ sao cho
    \[a^n + b^n = c^n\]
\end{theorem}

Và cú lừa có lẽ là lớn nhất thời đại: \textit{"Tôi đã tìm được chứng minh cho mệnh đề kỳ diệu này nhưng lề sách quá chật không thể viết được"}.

Vâng, cái chứng minh kỳ diệu mà ông nói đã khiến các nhà toán học thiên tài bế tắc trong suốt hơn 300 năm, sử dụng nhiều công cụ phức tạp không có ở thời Fermat và hoàn thiện bởi bài báo 200 trang của Andrew Wiles.

Nghĩa là 200 lề sách cũng không viết đủ chứng minh cho định lý cuối cùng của Fermat!!!

Phần này mình làm vì đam mê tìm hiểu lịch sử toán. Ở đây ghi lại cuộc đời và công trình của các nhà toán học lớn trên thế giới suốt chiều dài lịch sử.

Phần này lấy cảm hứng từ quyển \textit{Thiên tài và số phận} và \textit{Định lý cuối cùng của Fermat} của thầy Lê Quang Ánh, thông tin tham khảo dựa trên nhiều nguồn (chủ yếu là quyển \textit{Men of Mathematics} của E.T.Bell).

Tuy nhiên thông tin về cuộc đời của các nhà toán học đã có khá nhiều, mình sẽ trình bày theo cách hiểu của bản thân và đôi khi tập trung nhiều vào các công trình mức cơ sở.

Ngoại trừ phần lịch sử của nhà toán học, mình sẽ trình bày các định lý, khái niệm, ứng dụng của họ theo cách viết, cách trình bày của toán học hiện đại ngày nay để dễ tiếp cận.