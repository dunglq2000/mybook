\chapter{Hàm \foreignlanguage{german}{Möbius}}

\foreignlanguage{german}{August Ferdinand Möbius} là nhà toán học người Đức, đóng
góp nổi tiếng của ông là dải \foreignlanguage{german}{Möbius}. Tuy nhiên ở
đây chúng ta xem xét một hàm số học mang tên ông. Hàm \foreignlanguage{german}{Möbius}
đóng vai trò quan trọng trong việc tính các đại lượng liên quan tới số học.

\section{Hàm \foreignlanguage{german}{Möbius}}

\begin{definition}
    Hàm \foreignlanguage{german}{Möbius} của số nguyên dương $n$ được định nghĩa như sau:

    \begin{equation}
        \mu (n) = \begin{cases}
            1, \quad & \text{nếu}\, n = 1 \\
            (-1)^k, \quad & \text{nếu}\, n = p_1 p_2 \ldots p_k\,-\, p_i \text{ là số nguyên tố} \\
            0, \quad & \text{trong các trường hợp còn lại}
        \end{cases}
    \end{equation}
\end{definition}

Điều này có nghĩa là, nếu $n$ là tích của các số nguyên tố bậc 1 thì $\mu (n) = (-1)^k$
với $k$ là số lượng số nguyên tố trong tích. Như vậy, nếu tồn tại số nguyên tố
$p$ sao cho $p^2 \vert n$ thì $\mu(n) = 0$.

\section{Tính chất hàm \foreignlanguage{german}{Möbius}}

\begin{enumerate}
    \item Nếu $(n_1, n_2) = 1$ thì $\mu(n_1, n_2) = \mu(n_1) \mu(n_2)$
    \item $\displaystyle{\sum_{d \vert n} \mu(d) = 0}$ với $n = p_1 p_2 \ldots p_k$
\end{enumerate}

\begin{proof}
    Với tính chất 1, ta dễ thấy rằng do $n_1$ và $n_2$ nguyên tố cùng nhau nên trong
    cách phân tích thừa số nguyên tố của chúng sẽ chứa các số nguyên tố khác nhau. 
    Khi đó $\mu(n_1)$ và $\mu(n_2)$ không bị phụ thuộc nhau và có thể tách thành phép
    nhân như trên.

    Với tính chất 2, chúng ta lần lượt chọn $d$ là tổ hợp của 0, 1, 2, ..., $k$
    số nguyên tố.

    \begin{itemize}
        \item Nếu $d = 1$ thì $\mu(d) = 1$
        \item Nếu $d = p_i$ thì $\mu(d) = (-1)^1 = -1$ với $i = \overline{1, k}$
        \item Nếu $d = p_i p_j$ với $i \neq j$ thì $\mu (d) = (-1)^2 = 1$
        \item Tương tự như vậy, nếu $d$ là tích của $t$ số nguyên tố thì
        $\mu (d) = (-1)^t$
    \end{itemize}

    Ở mỗi trường hợp trên, do $d$ là tổ hợp của $t$ số nguyên tố ($0 \leq t \leq k$)
    nên số cách chọn số nguyên tố $p_i$ ở mỗi trường hợp là $C^t_k$. Cộng chúng lại

    \[\sum_{d \vert n} \mu(d) = 1 - C^1_k + C^2_k - \ldots + (-1)^k C^k_k = 0\]
    theo nhị thức Newton. Từ đó ta có điều phải chứng minh.
\end{proof}

\section{Công thức nghịch đảo \foreignlanguage{german}{Möbius}}

Giả sử ta có hai hàm $f$ và $g$ từ $\NN \to \ZZ$. Khi đó hai cách biểu diễn sau
là tương đương.

\begin{equation}
    f(n) = \sum_{d \vert n} g(d) \Leftrightarrow g(n) 
    = \sum_{d \vert n} f(d) \mu(\frac{n}{d})
\end{equation}

Nghĩa là nếu chúng ta có hai hàm số $f$ và $g$ thỏa phương trình đầu (biểu diễn
$f$ theo $g$) thì chúng ta cũng sẽ tìm được cách biểu diễn $g$ theo $f$.

\begin{proof}
    Với $d \vert n$, đặt $d' = \dfrac{n}{d} \Rightarrow d = \dfrac{n}{d'}$.

    Suy ra $f(d) \cdot \mu\Bigl(\dfrac{n}{d}\Bigr) = 
    f \Bigl(\dfrac{n}{d'}\Bigr) \cdot \mu(d')$.
    Sau đó lấy tổng lại thì

    \[\sum_{d \vert n} f(d) \cdot \mu \Bigl(\frac{n}{d}\Bigr) = 
    \sum_{d \vert n} f\Bigl(\frac{n}{d'}\Bigr) \cdot  \mu(d') = 
    \sum_{d \vert n} f\Bigl(\frac{n}{d}\Bigr) \cdot \mu(d)\]
    
    Ở đây lưu ý rằng nếu $d$ là ước của $n$ thì $d' = \dfrac{n}{d}$ cũng là
    ước của $n$. Do đó ta hoàn toàn có thể thay thế $d'$ bởi $d$ trong tổng trên.

    Vì $f(n) = \displaystyle{\sum_{d \vert n} g(d)}$ nên 
    \begin{equation}
        \label{eq:mebius1}
        \sum_{d \vert n} f\Bigl(\frac{n}{d}\Bigr) \cdot \mu(d)
    = \sum_{d \vert n} \mu(d) \sum_{d' \vert \frac{n}{d}} g(d')
    \end{equation}

    Dễ thấy rằng do $d \vert n$ và $d' \vert \dfrac{n}{d}$ nên tồn tại
    $k$, $l$ sao cho $kd = n$ và $ld' = \dfrac{n}{d}$. Khi đó $n = ldd'$ và $kd = n$.
    Suy ra $d' \vert n$ và $d \vert \dfrac{n}{d'}$.

    Tương tự như trên, ta có thể thay thế $d$ bởi $d'$ và ngược lại
    \[(\ref{eq:mebius1}) = \sum_{d' \vert n} g(d') \sum_{d \vert \frac{n}{d'}} \mu(d)\]
    mà $\displaystyle{\sum_{a \vert p} \mu(a) = 0}$ nếu $p \neq 1$ và bằng 1 với $p = 1$.
    (đã chứng minh ở trên), nên từ đây suy ra
    \[
        \sum_{d' \vert n} g(d') \sum_{d \vert \frac{n}{d'}} \mu(d)
        = \sum_{d' \vert n} g(d') \cdot 1 \,(\text{khi } n = d') = g(n)    
    \]
\end{proof}

Tương tự ta cũng có công thức nghịch đảo \foreignlanguage{german}{Möbius}
đối với phép nhân

\begin{equation}
    f(n) = \prod_{d \vert n} g(d) \Leftrightarrow 
    g(n) = \prod_{d \vert n} f(d)^{\mu\Bigl(\frac{n}{d}\Bigr)}
\end{equation}

\textbf{Liên hệ với hàm Euler}

Nếu ta chọn $f(n) = n$ và $g(n) = \phi(n)$ thì theo công thức nghịch đảo
\foreignlanguage{german}{Möbius} ta có
$\displaystyle{\phi(n) = \sum_{d \vert n} d \cdot \mu\Bigl(\frac{n}{d}\Bigr)}$
do ta đã biết $\displaystyle{\sum_{d \vert n} \phi(d) = n}$