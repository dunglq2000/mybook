\chapter{Mở đầu về số học}

Số học xuất hiện từ xa xưa, từ những bước đi đầu tiên của con
người. Tuy nhiên số học lại mang vẻ bí ẩn khó tưởng, sự phức
tạp vượt ra phạm vi số học. Nhà toán học vĩ đại Gauss từng nói
\textit{Toán học là vua của các môn khoa học, và số học là
nữ hoàng}. Hay một trong 23 bài toán thế kỷ của Hilbert về sự
phi mâu thuẫn của số học, người ta đã chứng minh được rằng
không thể chứng minh sự phi mâu thuẫn của số học chỉ bằng các
lý thuyết về số học.

\section{Phép chia Euclid}

Đây là nền tảng, cơ sở của số học. Từ khi biết tới phép chia
hai số nguyên, ta có thể tìm \textit{thương} và \textit{số dư}.
Nói theo toán học, nếu ta có 2 số nguyên dương $a$ và $b$, thì
tồn tại cặp số $q$, $r$ sao cho $a = qb + r$ với $0 \leq r < b$.

Khi đó, $a$ gọi là số bị chia, $b$ gọi là số chia, $q$ là thương
(q trong quotient) và $r$ là số dư (r trong remainder).

Đặc biệt là sự tồn tại của cặp số $q$ và $r$ là duy nhất. Thật
vậy, nếu ta giả sử tồn tại 2 cặp số $(q_1, r_1)$ và $(q_2, r_2)$ 
đều thỏa đẳng thức trên, nghĩa là
\[a = q_1 b + r_1, \quad a = q_2 b + r_2\]

Trừ 2 đẳng thức vế theo vế ta có $(q_1 - q_2) b + (r_1 - r_2) = 0$.
Tương đương $(r_2 - r_1) = (q_1 - q_2) b$, mà $0 \leq r_1, r_2 < b$
nên $-b < r_2 - r_1 < b$. Như vậy chỉ có thể xảy ra trường hợp
$r_2 - r_1 = 0$ hay $r_2 = r_1$, kéo theo $q_1 = q_2$.

\section{Thuật toán Euclid}

Dựa trên phép chia Euclid, ta có một thuật toán hiệu quả để tìm
ước chung lớn nhất giữa hai số $a$ và $b$.

Ký hiệu $\gcd(a, b)$ là ước chung lớn nhất của $a$ và $b$. Chúng 
ta thực hiện đệ quy như sau:
\[\gcd(a, b) = \begin{cases}
    a, \quad & \text{nếu}\,b = 0 \\
    \gcd(b, a \bmod b), \quad & \text{nếu}\,b \neq 0
\end{cases} 
    \]

Điểm quan trọng ở thuật toán Euclid là thuật toán chắc chắn sẽ dừng
sau một số hữu hạn bước, và kết quả sẽ là ước chung lớn nhất của 2
số $a$ và $b$.

\begin{proof}
    Đặt $r_0 = a$ và $r_1 = b$. Theo thuật chia Euclid
    ta có các số $q_0$ và $r_2$ sao cho $r_0 = r_1 q_0 + r_2$ với
    $0 \leq r_2 < r_1$. Thuật toán Euclid hoạt động như sau:
    \begin{align*}
        r_0 & = r_1 q_0 + r_2 \\
        r_1 & = r_2 q_1 + r_3 \\
        r_2 & = r_3 q_2 + r_4 \\
        \ldots & = \ldots \\
        r_i & = r_{i+1} q_i + r_{i+2} \\
        \ldots & = \ldots \\
        r_k & = r_{k+1} q_k + 0 \\
        r_{k+1} & = 0
    \end{align*}
    Ta thấy rằng ở mỗi bước, $r_{i+2}$ luôn nhỏ hơn $r_{i+1}$.
    Do đó cuối cùng sẽ bằng 0, và khi đó ta có ước chung lớn nhất.
\end{proof}

\section{Thuật toán Euclid mở rộng}

\begin{definition}[Phương trình Diophantos]
    Cho trước các số nguyên $a$, $b$ và $c$. Phương trình 
    Diophantus là phương trình có dạng
    \[ax + by = c\]
    với $x$, $y$ là các số nguyên.
\end{definition}

\begin{example}
    Giải phương trình $5x+3y = 1$.

    Ta có $y = \frac{1-5x}{3} = \frac{1-2x-3x}{3} = \frac{1-2x}{3} - x$.
    Như vậy nếu $y \in \ZZ$ thì $\frac{1-2x}{3} \in \ZZ$, nghĩa là
    $1-2x$ chia hết cho 3. Vậy $1-2x = 3k$ với $k \in \ZZ$.

    Tiếp tục, $1-2x = 3k$, suy ra $x = \frac{1-3k}{2} 
    = \frac{1-k-2k}{2} = \frac{1-k}{2} - k$. Do $x$ nguyên nên
    tương tự $\frac{1-k}{2}$ cũng nguyên, hay $1-k = 2t$, tương
    đương với $k = 1-2t$.

    Thay ngược lại ta có $x = \frac{1-3k}{2} = \frac{1-3(1-2t)}{2}
    = {-1+3t}$. Tiếp tục thay vào để tìm $y$ thì $y = \frac{1-5x}{3}
    = \frac{1-5(-1+3t)}{3} = 2 - 5t$.

    Như vậy nghiệm của phương trình là tất cả các nghiệm $(x, y)$
    mà $x = -1+3t$, $y = 2-5t$ với $t \in \ZZ$.
\end{example}

Ở đây chúng ta đã thực hiện phép chia có dư liên tiếp để tìm
nghiệm. Nói cách khác ta đã thực hiện thuật toán Euclid ở bên trên
để làm giảm độ phức tạp ở mỗi bước giải. Tổng quát ta có thuật toán 
Euclid mở rộng để tìm ước chung lớn nhất $\gcd(a, b)$ của hai 
số $a$, $b$, và \textbf{một} nghiệm của phương trình $ax + by = \gcd(a, b)$.

Ở ví dụ trên, ta thấy rằng $(-1, 2)$ là một nghiệm của phương
trình $5x + 3y = 1$. Khi đó ta có thể suy ra tất cả nghiệm
(họ nghiệm) của phương trình có dạng $(-1+3t, 2-5t)$ với $t \in \ZZ$.

\begin{algorithm}
    \caption{Thuật toán Euclid mở rộng}
    \begin{algorithmic}
        \Require $a, b \in \ZZ$
        \Ensure $\gcd(a, b)$, $x$, $y$ 
        \State $r_0 \gets a$, $r_1 \gets b$, $r_2 \gets 0$
        \State $x_0 \gets 1$, $x_1 \gets 0$, $x_2 \gets 0$
        \State $y_0 \gets 0$, $y_1 \gets 1$, $y_2 \gets 0$
        \While{$r_1 \neq 0$}
            \State $q \gets r_0 \;\text{div}\; r_1$
            \State $r_2 \gets r_0 - q * r_1$, $r_0 \gets r_1$, $r_1 \gets r_2$
            \State $x_2 \gets x_0 - q * x_1$, $x_0 \gets x_1$, $x_1 \gets x_2$
            \State $y_2 \gets y_0 - q * y_1$, $y_0 \gets y_1$, $y_1 \gets y_2$
        \EndWhile
        \State \Return $r_0$, $x_0$, $y_0$
    \end{algorithmic}
\end{algorithm}

Ở thuật toán trên, $r_0$, $r_1$ và $r_2$ hoạt động
như thuật toán Euclid chuẩn. Ở mỗi bước $q$ là thương của phép
chia hai số nguyên và ta sử dụng $q$ đó để tính $x_0$ và $y_0$
mới. Kết quả cuối cùng $(r_0, x_0, y_0)$ lần lượt là
ước chung lớn nhất, và hai số $x$, $y$ thỏa mãn $a x_0 + y b_0 = r_0$.

Tại sao chúng ta lại có $(x_0, x_1) = (1, 0)$ và
$(y_0, y_1) = (0, 1)$? Nói cách khác, làm sao biết thuật toán 
hoạt động đúng?

Mục đích của chúng ta là tìm các số $(x, y)$ sao cho $ax + by = \gcd(a, b)$.
Khi đó, dựa trên thuật toán Euclid cơ bản ở trên, ta xây dựng
dãy số $\{x_n\}$ và $\{y_n\}$ sao cho ở mọi bước thứ $n$ ta đều có

\begin{equation}\label{euclid:1}
    a x_n + b y_n = r_n
\end{equation}

Ta có $r_i = r_{i+1} q_i + r_{i+2}$. Từ $q_i$ ở mỗi bước ta tính
được

\begin{equation}
    x_i = x_{i+1} q_i + x_{i+2}, \quad y_i = y_{i+1} q_i + y_{i+2}
\end{equation}

Thay vào \ref{euclid:1} ta được
\begin{equation}
    a (x_{i+1} q_i + x_{i+2}) + b (y_{i+1} q_i + y_{i+2}) = r_i
\end{equation}

Tương đương với
\[(a x_{i+1} + b y_{i+1}) q_i + (a x_{i+2} + b x_{i+2}) = r_i\]

Mà $a x_{i+1} + b y_{i+1} = r_{i+1}$ và $a x_{i+2} + b y_{i+2}
= r_{i+2}$. Suy ra $r_{i+1} q_i + r_{i+2} = r_n$, đúng với thuật toán
Euclid chuẩn ban đầu. Nghĩa là thuật toán hoạt động đúng.
Bây giờ ta cần chọn $(x_0, x_1)$ và $(y_0, y_1)$ vì chúng ta
đã đặt $r_0 = a$ và $r_1 = b$. Ở bước thứ 0,
\[r_0 = a = a x_0 + b y_0\]
và ở bước thứ 1,
\[r_1 = b = a x_1 + b y_1\]

Dễ thấy ở bước 0 ta chọn $(1, 0)$ và ở bước 1 ta chọn $(0, 1)$ là được.