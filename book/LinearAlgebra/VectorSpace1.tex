\chapter{Tổng quan về không gian vector}

Ở chương này tên vector được in đậm chữ thường (ví dụ $\bm{x}$), tên ma trận được in đậm chữ hoa (ví dụ $\bm{M}$). Các số vô hướng (số thực) được viết không in đậm (ví dụ $x_1$, $x_2$).

\section{Hạng của ma trận}

\begin{definition}[Hạng của ma trận]
    
    Cho ma trận $\bm{M}_{m \times n}$ có $m$ hàng và $n$ cột. \textbf{Hạng} của ma trận $\bm{M}$ là cấp của ma trận vuông con lớn nhất của $\bm{M}$ có định thức khác 0.

    \textit{Ký hiệu}. Hạng (hay rank) của ma trận $\bm{M}$ được ký hiệu là $r = \rank(\bm{M})$

\end{definition}

\begin{remark}
    Nếu $r$ là hạng của ma trận $\bm{M}_{m \times n}$ thì $r \leq \min (m, n)$
\end{remark}

\section{Tổ hợp tuyến tính}

Xét tập hợp các vector $\{\bm{v}_1, \bm{v}_2, \ldots, \bm{v}_d\}$ trên $\RR$.

\begin{definition}[Tổ hợp tuyến tính]
Với vector $\bm{x}$ bất kì thuộc $\RR$, nếu tồn tại các số thực $\alpha_1, \alpha_2, \ldots, \alpha_d \in \RR$ sao cho
\[\bm{x} = \alpha_1 \bm{v}_1 + \alpha_2 \bm{v}_2 + \ldots + \alpha_d \bm{v}_d\]
thì $\bm{x}$ được gọi là \textbf{tổ hợp tuyến tính} của các vector $\bm{v}_i$, $i = 1, 2, \ldots, d$.
\end{definition}

Ta thấy rằng vector không $\bm{0}$ là tổ hợp tuyến tính của mọi tập các vector $\bm{v}_i$.

Bây giờ ta xét tổ hợp tuyến tính
\[\alpha_1 \bm{v}_1 + \alpha_2 \bm{v}_2 + \ldots + \alpha_d \bm{v}_d = \bm{0}\]

\begin{definition}[Độc lập tuyến tính]
    Tập hợp các vector $\bm{v}_1$, $\bm{v}_2$, ..., $\bm{v}_d$ được gọi \textbf{độc lập tuyến tính} nếu
    chỉ có duy nhất trường hợp $\alpha_1 = \alpha_2 = \ldots = \alpha_d = 0$ thỏa tổ hợp tuyến tính trên.    
\end{definition}

\begin{definition}[Phụ thuộc tuyến tính]
    Tập các vector là phụ thuộc tuyến tính nếu không độc lập tuyến tính.
    Nói cách khác tồn tại ít nhất một phần tử $\alpha_i \neq 0$.
\end{definition}

\section{Không gian vector}

Xét tập hợp các vector $\mathcal{V} \subset \RR^n$.

Ta định nghĩa hai phép tính cộng và nhân trên các vector này sao cho

\begin{itemize}[noitemsep]
    \item Phép cộng: Với mọi $\bm{x}, \bm{y} \in \mathcal{V}$ thì $\bm{x} + \bm{y} \in \mathcal{V}$
    \item Phân nhân vô hướng: Với mọi $\alpha \in \RR$ và $\bm{x} \in \mathcal{V}$ thì $\alpha \bm{x} \in \mathcal{V}$
\end{itemize}

Nói cách khác, phép cộng hai vector và phép nhân vô hướng một số thực với vector cho kết quả vẫn nằm trong không gian vector đó.

Đồng thời, phép cộng và phép nhân vô hướng phải thỏa mãn các tính chất sau

\begin{enumerate}[noitemsep]
    \item Tính giao hoán với phép cộng: với mọi $\bm{x}, \bm{y} \in \mathcal{V}$, $\bm{x} + \bm{y} = \bm{y} + \bm{x}$
    \item Tính kết hợp với phép cộng: với mọi $\bm{x}, \bm{y}, \bm{z} \in \mathcal{V}$, $\bm{x} + (\bm{y} + \bm{z}) = (\bm{x} + \bm{y}) + \bm{z}$
    \item Phần tử đơn vị của phép cộng: tồn tại vector không $\bm{0}$ sao cho với mọi $\bm{x} \in \mathcal{V}$, $\bm{0} + \bm{x} = \bm{x} + \bm{0} = \bm{x}$
    \item Phần tử đối của phép cộng: với mọi $\bm{x} \in \mathcal{V}$, tồn tại phần tử $\bm{x'} \in \mathcal{V}$ sao cho $\bm{x} + \bm{x'} = \bm{x} + \bm{x'} = \bm{0}$
    \item Phần tử đơn vị của phép nhân vô hướng: tồn tại số thực $1$ sao cho với mọi $\bm{x} \in \mathcal{V}$ thì $1 \cdot \bm{x} = \bm{x}$
    \item Tính kết hợp của phép nhân vô hướng: với mọi $\alpha, \beta \in \RR$, với mọi $\bm{x} \in \mathcal{V}$ thì $\alpha (\beta \bm{x}) = (\alpha \beta) \bm{x}$
    \item Tính phân phối giữa phép cộng và nhân: với mọi $\alpha \in \RR$, với mọi $\bm{x}, \bm{y} \in \mathcal{V}$ thì $\alpha (\bm{x} + \bm{y}) = \alpha \bm{x} + \alpha \bm{y}$
    \item Tính phân phối giữa phép nhân vô hướng: với mọi $\alpha, \beta \in \RR$, với mọi $\bm{x} \in \mathcal{V}$ thì $(\alpha + \beta) \bm{x} = \alpha \bm{x} + \beta \bm{x}$
\end{enumerate}

\section{Cơ sở và số chiều của không gian vector}

Nếu trong không gian vector $\mathcal{V}$ tồn tại các vector độc lập tuyến tính $\bm{v_1}$, $\bm{v_2}$, ..., $\bm{v_d}$
mà tất cả các vector trong $\mathcal{V}$ có thể biểu diễn dưới dạng tổ hợp tuyến tính của các vector $\bm{v_i}$ trên,
thì tập hợp các vector 
\[\{ \bm{v}_1, \bm{v}_2, \ldots, \bm{v}_d \}\]
được gọi là \textbf{cơ sở} của không gian vector $\mathcal{V}$.

Khi đó,
\[\bm{x} = \sum_{i=1}^{d} \alpha_i \bm{v}_i \quad \forall \bm{x} \in \mathcal{V}\]

Số lượng phần tử của tập hợp các vector đó (ở đây là $d$) gọi là \textbf{số chiều (dimension)} của không gian vector $\mathcal{V}$.
Ta ký hiệu $\text{dim} \mathcal{V} = d$.

Ta còn ký hiệu 
\[\mathcal{V} = \text{span} \{\bm{v}_1, \bm{v}_2, \ldots, \bm{v}_d\}\]
và nói là không gian vector $\mathcal{V}$ được span (hay được sinh) bởi các vector $\bm{v_i}$.

Ta thấy rằng có thể có nhiều cơ sở cho cùng một không gian vector.

\begin{theorem}
    Mọi cơ sở của không gian vector $\mathcal{V}$ đều có số phần tử bằng $\text{dim} \mathcal{V}$
\end{theorem}

Từ đó ta có điều kiện cần và đủ để một tập hợp vector là cơ sở của không gian vector.

Giả sử ta có $\bm{v}_1$, $\bm{v}_2$, ..., $\bm{v}_d$ là một cơ sở của không gian vector $\RR^n$.
Khi đó nếu hệ vector $\bm{w}_1$, $\bm{w}_2$, ..., $\bm{w}_d$ cũng là một hệ cơ sở khi và chỉ khi tồn 
tại ma trận khả nghịch $\bm{A}$ sao cho $\bm{W} = \bm{A} \cdot \bm{V}$. Ở đây $\bm{W}$ là ma trận với các hàng là các vector $\bm{w}_i$. Tương tự $\bm{V}$ là ma trận với các hàng là các vector $\bm{v}_i$.

\begin{proof}
    Ta viết các vector $\bm{v}_i$ dưới dạng $\RR^n$.

    \begin{align*}
        \bm{v}_1 & = (v_{11}, v_{12}, \ldots, v_{1n}) \\
        \bm{v}_2 & = (v_{21}, v_{22}, \ldots, v_{2n}) \\
        \ldots & = (\ldots, \ldots, \ldots, \ldots) \\
        \bm{v}_d & = (v_{d1}, v_{d2}, \ldots, v_{dn})
    \end{align*}

    Tương tự là các vector $\bm{w}_i$.

    \begin{align*}
        \bm{w}_1 & = (w_{11}, w_{12}, \ldots, w_{1n}) \\
        \bm{w}_2 & = (w_{21}, w_{22}, \ldots, w_{2n}) \\
        \ldots & = (\ldots, \ldots, \ldots, \ldots) \\
        \bm{w}_d & = (w_{d1}, w_{d2}, \ldots, w_{dn})
    \end{align*}

    Do $\bm{v}_i$ là một cơ sở của $\RR^n$, mọi vector trong $\RR^n$ được biểu diễn dưới dạng tổ hợp tuyến tính của các $\bm{v}_i$.

    Khi đó ta viết các $\bm{w}_i$ dưới dạng tổ hợp tuyến tính của $\bm{v_i}$.

    \begin{align*}
        \bm{w}_1 & = \alpha_{11} \bm{v}_1 + \alpha_{12} \bm{v}_2 + \ldots + \alpha_{1d} \bm{v}_d \\
        \bm{w}_2 & = \alpha_{21} \bm{v}_1 + \alpha_{22} \bm{v}_2 + \ldots + \alpha_{2d} \bm{v}_d \\
        \ldots & = \ldots \\
        \bm{w}_d & = \alpha_{d1} \bm{v}_1 + \alpha_{d2} \bm{v}_2 + \ldots + \alpha_{dd} \bm{v}_d
    \end{align*}

    Điều này tương đương với 

    \begin{align*}
        \begin{pmatrix}
            w_{11} & w_{12} & \ldots & w_{1n} \\
            w_{21} & w_{22} & \ldots & w_{2n} \\
            \ldots & \ldots & \ldots & \ldots \\
            w_{d1} & w_{d2} & \ldots & w_{dn}
        \end{pmatrix}
        = & \begin{pmatrix}
            \alpha_{11} & \alpha_{12} & \ldots & \alpha_{1d} \\
            \alpha_{21} & \alpha_{22} & \ldots & \alpha_{2d} \\
            \ldots & \ldots & \ldots & \ldots \\
            \alpha_{d1} & \alpha_{d2} & \ldots & \alpha_{dd}
        \end{pmatrix} \\
        \times & \begin{pmatrix}
            v_{11} & v_{12} & \ldots & v_{1n} \\ 
            v_{21} & v_{22} & \ldots & v_{2n} \\ 
            \ldots & \ldots & \ldots & \ldots \\ 
            v_{d1} & v_{d2} & \ldots & v_{dn}
        \end{pmatrix}
    \end{align*}
    
    Nếu $\bm{w}_i$ cũng là cơ sở của $\mathcal{V}$, thì các vector $\bm{v}_i$ cũng phải
    biểu diễn được dưới dạng tổ hợp tuyến tính của $\bm{w}_i$.
    Nói cách khác, ma trận $(\alpha_{ij})$ khả nghịch và ta có điều phải chứng minh.
\end{proof}



\section{Không gian vector con}

Cho không gian vector $\mathcal{V} \subset \RR^n$ với phép cộng hai vector
và phép nhân vô hướng. Một tập con $L$ của $\mathcal{V}$ được gọi
là không gian vector con nếu:

\begin{itemize}
    \item Với mọi $\bm{x}$, $\bm{y}$ thuộc $L$, $\bm{x} + \bm{y} \in L$
    \item Với mọi $\alpha \in \RR$, với mọi $\bm{x} \in L$, $\alpha \bm{x} \in L$
\end{itemize}

Nói cách khác, phép cộng và phép nhân vô hướng \textit{đóng} trong
không gian vector con.

\begin{remark}
    Trên $\RR^n$, hệ phương trình tuyến tính thuần nhất có thể sinh
    ra một không gian vector con của $\RR^n$.
\end{remark}

\begin{example}
    Xét hệ phương trình tuyến tính sau:

    \begin{equation}
        \begin{array}{cccccccc}
            x_1 & + & 3x_2 & + & 5x_3 & + & 7x_4 & = 0 \\
            2x_1 & & & + & 4x_3 & + & 2x_4 & = 0 \\
            3x_1 & + & 2x_2 & + & 8x_3 & + & 7x_4 & = 0
        \end{array}
    \end{equation}

    Biến đổi ma trận
    \[
        \begin{pmatrix}
            1 & 3 & 5 & 7 \\
            2 & 0 & 4 & 2 \\
            3 & 2 & 8 & 7
        \end{pmatrix} \sim \begin{pmatrix}
            1 & 3 & 5 & 7 \\
            0 & -6 & -6 & -12 \\
            0 & -7 & -7 & -14
        \end{pmatrix} \sim \begin{pmatrix}
            1 & 3 & 5 & 7 \\
            0 & 1 & 1 & 2 \\
            0 & 0 & 0 & 0
        \end{pmatrix}
    \]

    Như vậy hệ tương đương với
    \[x_1 + 3x_2 + 5x_3 + 7x_4 = 0, \quad x_2 + x_3 + 2x_4 = 0\]

    Ta chọn $x_3, x_4 \in \RR$ tự do, khi đó $x_1$ và $x_2$ được 
    biểu diễn theo $x_3$ và $x_4$ 
    \begin{equation}
        x_1 = -2x_3 - x_4, \quad x_2 = -x_3 - 2x_4
    \end{equation}

    Mọi vector trong không gian tuyến tính khi đó có dạng
    \begin{align*}
        (x_1, x_2, x_3, x_4) & = (-2x_3 - x_4, -x_3 - 2x_4, x_3, x_4)
        \\ & = x_3 \cdot (-2, -1, 1, 0) + x_4 \cdot (-1, -2, 0, 1) 
    \end{align*}

    Ở đây ta thấy $x_3, x_4$ nhận giá trị tùy ý trong $\RR$, và
    mọi vector trong không gian nghiệm là tổ hợp tuyến tính của
    hai vector $(-2, -1, 1, 0)$ và $(-1, -2, 0, 1)$. Suy ra hai vector
    này là cơ sở của không gian nghiệm, và $\dim \mathcal{V} = 2$.

\end{example}

\section{Không gian Euclide}

Trên không gian vector $\mathcal{V}$ chúng ta bổ sung thêm một phép toán là tích vô hướng (dot product, tích chấm) của hai vector.

Giả sử với hai vector $\bm{x} = (x_1, x_2, \ldots, x_n)$ và $\bm{y} = (y_1, y_2, \ldots, y_n)$. Khi đó tích vô hướng của $\bm{x}$ và $\bm{y}$ là

\begin{equation}
	\bm{x} \cdot \bm{y} = x_1 y_1 + x_2 y_2 + \ldots + x_n y_n
\end{equation}

Một số sách ký hiệu tích vô hướng của hai vector $\bm{x}$ và $\bm{y}$ là $\langle \bm{x}, \bm{y} \rangle$. Trong quyển sách này mình sẽ dùng ký hiệu $\bm{x} \cdot \bm{y}$ như trên.

Không gian vector có phép toán tích vô hướng được gọi là không gian Euclide. Khi $\bm{x} = \bm{y}$ thì căn bậc hai của kết quả tích vô hướng được gọi là \textbf{chuẩn Euclide} (Euclidean norm) và được ký hiệu

\begin{equation}
	\lVert \bm{x} \rVert = \sqrt{\bm{x} \cdot \bm{x}} = \sqrt{x_1^2 + x_2^2 + \ldots + x_n^2}
\end{equation}

Như vậy ta có thể viết $\lVert \bm{x} \rVert^2 = \bm{x}^2$.

Nếu biểu diễn hai vector $\bm{x}$ và $\bm{y}$ biểu diễn điểm $A$ và $B$ trên hệ tọa độ $n$ chiều với tâm $O$, đặt $\varphi$ là góc giữa hai vector $\overrightarrow{OA}$ và $\overrightarrow{OB}$. Khi đó 

\begin{equation}
	\cos \varphi = \frac{\overrightarrow{OA} \cdot \overrightarrow{OB}}{\lVert OA \rVert \cdot \lVert OB \rVert} = \frac{\bm{x} \cdot \bm{y}}{\lVert \bm{x} \rVert \cdot \lVert \bm{y} \rVert}
\end{equation}

Ta có thể thấy định nghĩa này là trường hợp tổng quát của định lý hàm số cosin học ở phổ thông. Nếu ta có $\bm{x} \cdot \bm{y} = 0$ thì ta nói hai vector $\bm{x}$ và $\bm{y}$ vuông góc nhau, hay \textbf{trực giao} (orthogonal).

\begin{theorem}[Bất đẳng thức Cauchy-Schwarz]
	Với hai vector $\bm{x}$ và $\bm{y}$ bất kì ta luôn có
	
	\begin{equation}
		\lVert \bm{x} \rVert \cdot \lVert \bm{y} \rVert \geq \lvert \bm{x} \cdot \bm{y} \rvert
	\end{equation}
	
	Nghĩa là tích độ dài của hai vector bất kì trong cùng không gian Euclide lớn hơn hoặc bằng tích vô hướng giữa chúng. Dấu bằng xảy ra khi và chỉ khi $\dfrac{x_1}{y_1} = \dfrac{x_2}{y_2} = \ldots = \dfrac{x_n}{y_n}$. Nói cách khác là hai vector cùng phương.
\end{theorem}

\begin{proof}
	Với mọi số thực $t$, ta luôn có 
	
	\[0 \leq \lVert \bm{x} - t \bm{y} \rVert^2 = \bm{x}^2 - 2 t \bm{x} \cdot \bm{y} + t^2 \bm{y}^2 = \lVert \bm{x} \rVert^2 - 2 t \bm{x} \cdot \bm{y} + t^2 \lVert \bm{y} \rVert^2 \]
	
	Nếu xem biểu thức trên là đa thức bậc 2 theo $t$, để đa thức lớn hơn hoặc bằng 0 với mọi $t \in \RR$ thì ta phải có $\Delta' \leq 0$ và $\lVert \bm{y} \rVert^2 > 0$ (luôn đúng). Ta có
	\[\Delta' = (\bm{x} \cdot \bm{y})^2 - \lVert \bm{x} \rVert^2 \cdot \lVert \bm{y} \rVert^2 \leq 0 \]
	Tương đương với $\lvert \bm{x} \cdot \bm{y} \rvert \leq \lVert \bm{x} \rVert \cdot \lVert \bm{y} \rVert$ (điều phải chứng minh).
\end{proof}

\section{Hệ cơ sở trực giao}

Cho không gian Euclide $\mathcal{V}$ và một cơ sở của nó là $\bm{v}_1$, $\bm{v}_2$, ..., $\bm{v}_d$. Thuật toán trực giao Gram-Schmidt là thuật toán biến đổi cơ sở trên thành một cơ sở mới, trong đó các vector đều trực giao nhau.

\begin{algorithm}
	\caption{Thuật toán trực giao Gram-Schmidt}
	\begin{algorithmic}
		\Require $\bm{v}_1$, ..., $\bm{v}_d$ trong $\RR^n$
		\Ensure $\bm{u}_1$, ..., $\bm{u}_d$ trong $\RR^n$ mà $\bm{u}_i \cdot \bm{u}_j = 0$ với mọi $i \neq j$
		\State $\bm{u}_1 \gets \bm{v}_1$
		\For{$i = 2 \to d$}
			\State $\bm{w} = \bm{v}_i$
			\For{$j = i-1 \to 1$}
			\State $\mu_{i,j} = (\bm{v}_i \cdot \bm{u}_j) / (\bm{u}_i \cdot \bm{u}_j)$
			\State $\bm{w} \gets \bm{w} - \mu_{i, j} \bm{u}_j$
			\EndFor
			\State $\bm{u}_i \gets \bm{w}$
		\EndFor
		\State \Return $\bm{u}_1$, ..., $\bm{u}_d$
	\end{algorithmic}
\end{algorithm}

Nói cách khác, với $\bm{u}_1 = \bm{v}_1$, với mỗi $i = 2, 3, \ldots, d$ ta tính vector $\bm{u}_i$ với công thức

\begin{equation}
	\bm{u}_i = \bm{v}_i - \sum_{j=1}^{i-1} \mu_{i,j} \bm{u}_j
\end{equation}

Ở đây $\mu_{i,j} = \dfrac{\bm{v}_i \cdot \bm{u}_j}{\bm{u}_i \cdot \bm{u}_j}$ là hệ số trước $\bm{u}_j$.

\begin{example}
	Xét cơ sở $\bm{v}_1 = (2, -2, 4)$, $\bm{v}_2 = (1, -1, 0)$ và $\bm{v}_3 = (5, -3, 3)$ của $\RR^3$.
	
	Đặt $\bm{u}_1 = \bm{v}_1 = (2, -2, 4)$.
	
	Ta có $\mu_{2,1} = \dfrac{\bm{v}_2 \cdot \bm{u}_1}{\bm{u}_1 \cdot \bm{u}_1} = \dfrac{1 \cdot 2 + (-1) \cdot (-2) + 0 \cdot 4}{2^2 + (-2)^2 + 4^2} = \dfrac{4}{24} = \dfrac{1}{6}$.
	
	Suy ra \[\bm{u}_2 = \bm{v_2} - \mu_{2, 1} \bm{u}_1 = (1, -1, 0) - \dfrac{1}{6} \cdot (2, -2, 4) = \Bigl(\dfrac{2}{3}, \dfrac{-2}{3}, \dfrac{-2}{3}\Bigr)\]
	
	Tương tự $\mu_{3, 1} = \dfrac{\bm{v}_3 \cdot \bm{u}_1}{\bm{u}_1 \cdot \bm{u}_1} = \dfrac{5 \cdot 2 + (-3) \cdot (-2) + 3 \cdot 4}{2^2 + (-2)^2 + 4^2} = \dfrac{28}{24} = \dfrac{7}{6}$.
	
	Tiếp theo $\mu_{3,2} = \dfrac{\bm{v}_3 \cdot \bm{u}_2}{\bm{u}_2 \cdot \bm{u}_2} = \dfrac{5 \cdot \dfrac{2}{3} + (-3) \cdot \dfrac{-2}{3} + 3 \cdot \dfrac{-2}{3}}{ \Bigl(\dfrac{2}{3}\Bigr)^2 + \Bigl(\dfrac{-2}{3}\Bigr)^2 + \Bigl(\dfrac{-2}{3}\Bigr)^2 } = \dfrac{5}{2}$.
	
	\begin{align*}
		\Rightarrow \quad \bm{u}_3 = & \bm{v}_3 - \mu_{3,1} \bm{u}_1 - \mu_{3,2} \bm{u}_2 \\ = & (5, -3, 3) - \dfrac{7}{6} \cdot (2, -2, 4) - \dfrac{5}{2} \cdot \Big(\dfrac{2}{3}, \dfrac{-2}{3}, \dfrac{-2}{3}\Big) \\ = & (1, 1, 0)
	\end{align*}
	
	Ta có thể kiếm chứng rằng $\bm{u}_1 \cdot \bm{u}_2 = 2 \cdot \dfrac{2}{3} + (-2) \cdot \dfrac{-2}{3} + 4 \cdot \dfrac{-2}{3} = 0$. Tương tự với $\bm{u}_1 \cdot \bm{u}_3 = 0$ và $\bm{u}_2 \cdot \bm{u}_3 = 0$. Thêm nữa các vector này cũng độc lập tuyến tính nên cũng là một hệ cơ sở của $\RR^3$.
	
	Như vậy các vector $\bm{u}_1$, $\bm{u}_2$, $\bm{u}_3$ là một cơ sở trực giao của $\RR^3$.
\end{example}