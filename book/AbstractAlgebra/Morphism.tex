\chapter{Group homomorphism}

Đồng cấu nhóm (group homomorphism) đóng vai trò quan trọng trong lý thuyết nhóm.
Nhờ nó chúng ta có thể chuyển việc tính toán trên nhóm này sang nhóm khác (thường
là dễ tính toán hơn).

\section{Đồng cấu nhóm}

\begin{definition}[Homomorphism]
    Xét hai nhóm $(G, \star)$ và $(H, *)$ và một ánh xạ $f: G \to H$.
    Ánh xạ $f$ được gọi là \textbf{homomorphism} nếu với mọi $g_1$, $g_2$ thuộc
    $G$ ta có $f(g_1 \star g_2) = f(g_1) * f(g_2)$.
\end{definition}

Do $g_1$, $g_2$ là các phần tử thuộc $G$ nên toán tử giữa chúng là $\star$. Trong
khi đó $f(g_1)$, $f(g_2)$ là các phần tử thuộc $H$ nên toán tử giữa chúng là $*$.

Từ định nghĩa chúng ta có thể rút ra một số nhận xét sau:

\begin{remark} Chúng ta có một số nhận xét quan trọng sau
    \begin{enumerate}
        \item Gọi $e_G$ là phần tử đơn vị của $G$ và $e_H$ là phần tử đơn
        vị của $H$. Khi đó $f(e_G) = e_H$
        \item Với mọi phần tử $g \in G$, nếu $g^{-1}$ là nghịch đảo của nó trong
        $G$ thì $f(g^{-1}) = f(g)^{-1}$
    \end{enumerate}
\end{remark}

\begin{proof}
    Việc chứng minh không quá phức tạp.
    \begin{enumerate}
        \item Nếu $e_G$ là phần tử đơn vị của $G$ thì với mọi $g \in G$ ta có
        $g \star e_G = e_G \star g = g$. Ta lấy $f$ cả 3 vế và theo định nghĩa
        homomorphism thu được
        $f(g \star e_G) = f(e_G \star g) = f(g) \Rightarrow f(g) * f(e_G) 
        = f(e_G) * f(g) = f(g)$. Đẳng thức trên đúng với mọi $g \in G$ nên đúng với
        mọi $f(g)$, suy ra $f(e_G)$ là phần tử đơn vị trong nhóm $(H, *)$ và
        do đó $f(e_G) = e_H$
        \item Từ việc tìm ra phần tử đơn vị, ta cũng chứng minh được
        tính chất nghịch đảo trên.
    \end{enumerate}
\end{proof}

\section{Các loại homomorphism}

Tương tự như ánh xạ, chúng ta có các loại homomorphism sau

\begin{definition}[Monomorphism]
    Ánh xạ được gọi là đơn cấu (monomorphism) nếu nó là ánh xạ one-to-one (đơn ánh).
    Nói cách khác, với mọi $g_1 \neq g_2$ và $g_1$, $g_2 \in G$, thì
    $f(g_1) \neq f(g_2)$
\end{definition}

\begin{definition}[Epimorphism]
    Ánh xạ được gọi là toàn cấu (epimorphism) nếu nó là ánh xạ onto (toàn ánh).
    Nói cách khác, với mọi $h \in H$ thì tồn tại $g \in G$ mà $f(g) = h$.
\end{definition}

\begin{definition}[Isomorphism]
    Ánh xạ được gọi là đẳng cấu (isomorphism) nếu nó là ánh xạ one-to-one 
    và onto (song ánh).
    Nói cách khác, ánh xạ này vừa là đơn cấu, vừa là toàn cấu.
\end{definition}

\section{Hạt nhân và ảnh}

Xét một homomorphism $f$ từ nhóm $(G, \star)$ tới nhóm $(H, *)$. Ta nói

\begin{definition}[Kernel]
    Hạt nhân (kernel) của $f$ là tập hợp các phần tử của $G$ cho ảnh là $e_H$, ký hiệu
    là $\Ker f$. Nói cách khác
    \begin{equation}
        \Ker f = \{ g \in G, f(g) = e_H \}
    \end{equation}
    Như vậy $\Ker f$ là tập con của $G$.
\end{definition}

\begin{remark}
    $K = \Ker f$ là normal subgroup của $G$.

    Để chứng minh, ta thấy rằng theo định nghĩa homomorphism,
    với $g_1, g_2 \in K$ thì $f(g_1) = f(g_2) = e_H$.
    
    Ta có $f(g_1 \star g_2) = f(g_1) * f(g_2) = e_H * e_H = e_H$. Như vậy
    $g_1 \star g_2 \in K$ nên $K$ là nhóm con của $G$.
    
    Tiếp theo để chứng minh $K$ là normal subgroup, ta chứng minh
    $g K g^{-1} = K$ với mọi $g \in G$.

    Do $g K g^{-1} = \{ g \star k \star g^{-1} : k \in K \}$, lấy $f$ mỗi phần 
    tử bên trong ta có 
    \[f(g \star k \star g^{-1}) = f(g) * f(k) * f(g^{-1}) = 
    f(g) * e_H * f(g^{-1}) = f(g) * f(g^{-1})\]
    , mà theo tính chất của homomorphism thì $f(g^{-1}) = f(g)^{-1}$ nên 
    $f(g \star k \star g^{-1}) = f(g) * f(g)^{-1} = e_H$ nên 
    $g \star k \star g^{-1} \in K$ với mọi
    $g \in G$, với mọi $k \in K$. Do đó $g K g^{-1} = K$ và ta có điều phải
    chứng minh.
\end{remark}

\begin{definition}[Image]
    Ảnh (image) của $f$ là tập hợp tất cả giá trị nhận được khi biến các phần
    tử thuộc $G$ thành phần tử thuộc $H$. Nói cách khác
    \begin{equation}
        \Img f = \{ f(g), g \in G \}
    \end{equation}
    Như vậy $\Img f$ là tập con của $H$.
\end{definition}

Dựa trên hai khái niệm này, chúng ta có một định lý quan trọng trong lý
thuyết nhóm là \textbf{Định lý thứ nhất về sự đẳng cấu} (First isomorphism theorem).

\begin{theorem}[First isomorphism theorem]
    Với hai nhóm $(G, \star)$ và $(H, *)$. Xét homomorphism $f: G \to H$. Khi 
    đó $\Img f$ đẳng cấu (isomorphism) với nhóm thương $G / \Ker f$.
\end{theorem}

\begin{proof}
    Gọi $G$, $H$ là hai nhóm và homomorphism $f: G \to H$.
    Đặt $K = \Ker f$. Ta xét biến đổi
    \[\theta:\,\Img f \to G / K, f(g) \to g K \]
    với $g \in G$.

    Ta cần chứng minh biến đổi này là ánh xạ xác định (well-defined, nghĩa là 
    tuân theo quy tắc ánh ánh xạ, mỗi phần tử tập nguồn biến thành \textbf{một và chỉ một}
    phần tử tập đích), là homomorphism, là đơn ánh và là toàn ánh.

    Đầu tiên ta chứng minh ánh xạ xác định. Giả sử ta có $g_1 K = g_2 K$, do $g_1$ và
    $g_2$ thuộc cùng coset nên $g_1^{-1} g_2 \in K$, hay $f(g_1^{-1} g_2) = e_H$.
    Với $f$ là homomorphism, ta có 
    \[f(g_1^{-1} g_2) = f(g_1^{-1}) f(g_2) =
    f(g_1)^{-1} f(g_2) = e_H\]
    Suy ra $f(g_1) = f(g_2)$. Như vậy nếu $f(g_1) = f(g_2)$
    thì $\theta (f(g_1)) = \theta (f(g_2))$.

    Tiếp theo ta chứng minh $\theta$ là homomorphism. Do $K$ là normal subgroup của
    $G$ nên với mọi $g_1$, $g_2$ thuộc $G$ thì $g_1 g_2 K = (g_1 K) (g_2 K)$.

    Do $f(g_1 g_2) = f(g_1) f(g_2)$ nên 
    \[ \theta (f(g_1 g_2)) = g_1 g_2 K = (g_1 K) (g_2 K) = \theta (f(g_1)) 
    \theta (f(g_2)) \]
    Suy ra $\theta$ là homomorphism.

    Dễ thấy với mọi $g \in G$ ta đều tìm được $f(g)$ và $g K$ tương ứng. Do đó $\theta$
    là toàn ánh.

    Để chứng minh $\theta$ là đơn ánh, giả sử $g_1 K = g_2 K$ ta có $g_1^{-1} g_2 \in K$
    nên $f(g_1^{-1} g_2) = e_H$. Suy ra $f(g_1^{-1}) f(g_2) = e_H \Rightarrow 
    f(g_1)^{-1} f(g_2) = e_H \Rightarrow f(g_1) = f(g_2)$. Như vậy $\theta$ là đơn ánh.

    Kết luận, $\theta$ là song ánh. Định lý thứ nhất về sự đẳng cấu được chứng minh.
\end{proof}