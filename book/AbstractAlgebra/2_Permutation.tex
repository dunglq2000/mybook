\chapter{Nhóm hoán vị}

Nhóm hoán vị đóng vai trò quan trọng trong lý thuyết nhóm cũng như nhiều hướng
khác của toán học.

\section{Nhóm hoán vị}

Xét tập hợp $\{ 1, 2, \ldots, n \}$. Ta gọi $\mathcal{S}_n$ là tập tất cả hoán vị 
của tập hợp trên. Như vậy $\mathcal{S}_n$ có $n!$ phần tử.

Ta ký hiệu mỗi phần tử của $\mathcal{S}_n$ là $\sigma =
(\sigma(1), \sigma(2), \ldots, \sigma(n))$. Như vậy, lấy hoán vị gốc là $(1, 2, 
\ldots n)$, mỗi hoán vị đều có thể được biểu diễn bằng hai hàng như sau

\begin{equation}
    \sigma = \begin{pmatrix}
        1 & 2 & \ldots & n \\
        \sigma(1) & \sigma(2) & \ldots & \sigma(n)
    \end{pmatrix}
\end{equation}

Điều đó có nghĩa là hai cách biểu diễn sau là tương đương

\begin{equation}
    \sigma = (\sigma(1), \sigma(2), \ldots, \sigma(n))
    = \begin{pmatrix}
        1 & 2 & \ldots & n \\
        \sigma(1) & \sigma(2) & \ldots & \sigma(n)
    \end{pmatrix}
\end{equation}

Ta định nghĩa toán tử trên $\mathcal{S}_n$. Với hai hoán vị $\sigma$ và $\tau$,
hoán vị $\sigma \star \tau$ là vị trí của $\sigma$ theo $\tau$. Nói cách
khác, nếu $\sigma = (\sigma_1, \sigma_2, \ldots, \sigma_n)$ và 
$\tau = (\tau_1, \tau_2, \ldots, \tau_n)$ thì $\sigma \star \tau =
(\sigma_{\tau_1}, \sigma_{\tau_2}, \ldots, \sigma_{\tau_n})$.

Nhóm $\mathcal{S}_n$ và toán tử như trên tạo thành một nhóm và được gọi là
\textbf{nhóm hoán vị}.

\begin{example}
    Xét nhóm hoán vị $\mathcal{S}_5$. 
    
    Gọi $x = (4, 3, 1, 2, 5)$ và
    $y = (5, 1, 4, 3, 2)$. Khi đó, đặt $z = x \star y$ thì
    \begin{align*}
        & z_1 = x_{y_1} = x_5 = 5, \\
        & z_2 = x_{y_2} = x_1 = 4, \\
        & z_3 = x_{y_3} = x_4 = 2, \\
        & z_4 = x_{y_4} = x_3 = 1, \\ 
        & z_5 = x_{y_5} = x_2 = 3
    \end{align*}

    Như vậy $z = x \star y = (5, 4, 2, 1, 3)$.
\end{example}

\begin{remark}
    Trong một hoán vị, khi biểu diễn trên hai hàng thì thứ tự viết không quan trọng,
    miễn là đảm bảo $i$ tương ứng với $\sigma(i)$ trên từng cột.
\end{remark}

\begin{example}
    Xét hoán vị $\sigma = (4, 3, 1, 2, 5)$ thuộc $\mathcal{S}_5$.

    Ta có $\sigma(1) = 4$, $\sigma(2) = 3$, $\sigma(3) = 1$, $\sigma(4) = 2$
    và $\sigma(5) = 5$. Như vậy

    \begin{equation*}
        \sigma = 
        \begin{pmatrix}
            1 & 2 & 3 & 4 & 5 \\
            4 & 3 & 1 & 2 & 5
        \end{pmatrix} = 
        \begin{pmatrix}
            3 & 4 & 5 & 1 & 2 \\
            1 & 2 & 5 & 4 & 3
        \end{pmatrix}
    \end{equation*}
\end{example}