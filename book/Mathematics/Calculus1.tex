\chapter{Giới hạn}

\section{Giới hạn của dãy số}

\begin{definition}[Giới hạn hữu hạn của dãy số]
Cho dãy số $\{a_n\}$. Ta nói dãy $\{a_n\}$ có giới hạn hữu hạn $L$ nếu với mọi 
$\varepsilon > 0$, tồn tại $n_0 \in \NN$ sao cho với mọi $n \geq n_0$ thì 
\[| a_{n} - L | < \varepsilon \]

Ký hiệu: $\displaystyle{\lim_{n \to \infty} a_n = L}$

\end{definition}

Nói cách khác, trên trục số với điểm $L$, nếu ta chọn một đường tròn bán kính $\varepsilon$ tùy ý,
thì mọi số hạng của dãy số kể từ số hạng $n_0$ nào đó trở đi đều nằm trong đường tròn này.
Thông thường $\varepsilon$ rất nhỏ.

\begin{example}
    Xét dãy số cho bởi công thức $a_n = \frac{1}{n}$.

    Ta chứng minh dãy số có giới hạn hữu hạn là 0.

    Với mọi $\varepsilon > 0$ tùy ý, ta cần chứng minh tồn tại
    số $n_0 \geq 1$ sao cho với mọi $n \geq n_0$ thì $| a_n - 0 | < \varepsilon$.

    Hay nói cách khác $a_{n_0} < \varepsilon$.

    Tương đương với $\frac{1}{n_0} < \varepsilon \Leftrightarrow n_0 > \frac{1}{\varepsilon}$

    Vậy ta chỉ cần chọn $n_0$ thỏa bất đẳng thức trên (luôn tìm được).

    Kết luận: $\displaystyle{\lim_{n \to \infty} a_n = 0}$
\end{example}

\begin{definition}[Dãy số có giới hạn vô cực]
    Cho dãy số $\{a_n\}$. Ta nói dãy số có giới hạn ở dương vô cực
    nếu với mọi $M > 0$, tồn tại $n_0 \in \NN$ sao cho với mọi $n \geq n_0$
    thì $a_n > M$.
\end{definition}

Nói cách khác, nếu ta chọn một số $M$ rất lớn bất kì, thì mọi số hạng
của dãy số kể từ một số hạng nào đó trở đi luôn lớn hơn $M$.

Định nghĩa về dãy số có giới hạn ở âm vô cực cũng tương tự.

\section{Giới hạn của hàm số}

Định nghĩa của hàm số theo kiểu Cauchy (hay còn được gọi là
ngôn ngữ $\delta-\varepsilon$) là kiểu định nghĩa phổ biến được
giảng dạy trong nhà trường.

\begin{definition}[Giới hạn hữu hạn của hàm số]
    Xét hàm số $f(x)$. Ta nói hàm số có giới hạn hữu hạn $L$
    khi $x$ tiến tới $x_0$, nếu với mọi $\varepsilon > 0$, tồn tại 
    $\delta > 0$ sao cho với mọi $x$ mà $| x - x_0 | < \delta$ thì
    $|f(x) - L| < \varepsilon$.

    Ký hiệu: $\displaystyle{\lim_{x \to x_0} f(x) = L}$
\end{definition}

Một cách hình ảnh, tương tự như giới hạn dãy số, lần này ta nhìn trên 2 trục 
của mặt phẳng tọa độ $Oxy$. Với mọi quả cầu bán kính $\varepsilon$ tâm $L$ (dành cho $f(x)$)
ta luôn chọn được quả cầu bán kính $\delta$ tâm $x_0$ (dành cho $x$). Lúc này khi $x$
nằm trong quả cầu tâm $x_0$ bán kính $\delta$ thì $f(x)$ tương ứng sẽ nằm trong quả cầu
tâm $L$ bán kính $\varepsilon$.

Ta có thể thấy ở đây $x$ tiến về $x_0$ (khá giống định nghĩa giới hạn hàm số)
và $f(x)$ tương ứng tiến về $L$.

Tương tự ta cũng có giới hạn hàm số ở vô cực

\begin{definition}[Giới hạn hàm số ở vô cực]
    Với hàm số $f(x)$, ta nói hàm số có giới hạn tại dương vô cực
    khi $x$ tiến về $x_0$ nếu với mọi $M > 0$, tồn tại $\delta > 0$ sao cho với mọi $x$ mà $|x - x_0| < \delta$ 
    thì $f(x) > M$

    Ký hiệu: $\displaystyle{\lim_{x \to x_0} f(x) = +\infty}$
\end{definition}

\begin{definition}[Giới hạn một bên]
    Ta nói hàm số $f(x)$ có giới hạn phải $L$ tại $x_0$ khi $x$ tiến về bên phải $x_0$
    nếu với mọi $\varepsilon > 0$, tồn tại $\delta > 0$ sao cho với mọi $0 < x - x_0 < \delta$ 
    thì $|f(x) - L| < \varepsilon$.

    Ký hiệu: $\displaystyle{\lim_{x \to x_0^+} f(x) = L}$
\end{definition}

Nghĩa là chúng ta chỉ xét giới hạn khi $x$ tiến tới $x_0$ từ bên phải $x > x_0$.
Tương tự cho giới hạn trái.

Lưu ý rằng trong nhiều trường hợp, mặc dù cùng tiến tới $x_0$ nhưng
giới hạn trái và giới hạn phải có thể không bằng nhau.

\begin{example}
    Xét hàm số $y = \frac{1}{x}$. Ta thấy hàm số không xác định
    tại $x = 0$, và giới hạn trái và phải khác nhau:
    \[\lim_{x \to 0^+} = +\infty, \quad \lim_{x \to 0^-} = -\infty\]
\end{example}

\section{Tính liên tục của hàm số}

Cho hàm số $f(x)$ xác định trên miền $D$ và $x_0$ là một điểm thuộc $D$.

\begin{definition}[Hàm số liên tục tại một điểm]
    Ta nói hàm số $f(x)$ liên tục tại $x_0$ nếu
    \[\lim_{x \to x_0} f(x) = f(x_0)\]
\end{definition}

Định nghĩa tương tự cho liên tục trái và liên tục phải (ta lấy giới hạn một bên).

Như vậy, có 3 khả năng hàm số không liên tục tại một điểm.

\begin{enumerate}[noitemsep]
    \item Hàm số không xác định tại $x_0$
    \item Hàm số xác định tại $x_0$ nhưng giới hạn tại đó không bằng $f(x_0)$
    \item Giới hạn trái và giới hạn phải không bằng nhau
\end{enumerate}

Nếu hàm số không liên tục tại $x_0$ ta gọi hàm số bị \textbf{gián đoạn} tại $x_0$.

\section{Đạo hàm}

\begin{definition}[Đạo hàm]
    Cho hàm số $f(x)$ xác định trên miền $D$ và $x_0$ là điểm thuộc $D$.
    Ta nói hàm số $f(x)$ có đạo hàm tại $x_0$ (hoặc khả vi tại $x_0$) nếu
    tồn tại giới hạn hữu hạn
    \[\lim_{x \to x_0}\frac{f(x) - f(x_0)}{x - x_0}\]
\end{definition}

\begin{example}
    Xét hàm số $f(x) = x^2 + 1$ trên $\RR$. Tìm đạo hàm tại $x_0 \in \RR$.

    Ta có $f(x)-f(x_0) = x^2 + 1 - (x_0^2 + 1) = (x - x_0) (x + x_0)$.

    Khi đó $\frac{f(x)-f(x_0)}{x-x_0} = x + x_0$ nên ta có
    \[ \lim_{x \to x_0} (x + x_0) = 2 x_0 \]
\end{example}

Nếu hàm số khả vi trên mọi điểm thuộc khoảng (đoạn) nào đó thì
ta nói hàm số khả vi trên khoảng (đoạn) đó và ký hiệu là $f'(x)$.

Với ví dụ trên, ta thấy giới hạn tồn tại với mọi $x_0 \in \RR$ nên ta có thể
thay $x_0$ bởi $x$ và có $f'(x) = 2x$ với $f(x) = x^2 + 1$.

\begin{remark}
    Từ định nghĩa ta thấy rằng nếu $f(x)$ khả vi tại $x_0$ thì nó cũng liên tục tại $x_0$.
    Lưu ý là chiều ngược lại không đúng. Ví dụ với hàm số $y = \lvert x \rvert$, 
    hàm số liên tục tại $x=0$ nhưng giới hạn (đạo hàm) phải là 1, 
    còn giới hạn (đạo hàm) trái là -1.
\end{remark}

Về mặt hình ảnh, khi hàm số khả vi tại một điểm thì đồ thị sẽ "trơn",
không gấp khúc tại điểm đó.

\section{Ý nghĩa cơ học của đạo hàm}

Xét một chất điểm (vật lý) chuyển động. Quãng đường chuyển động của chất điểm được 
biểu diễn bởi hàm số theo thời gian. Ta đã biết vận tốc là đặc trưng chuyển động của 
chất điểm trong một đơn vị thời gian (phản ánh chất điểm chuyển động nhanh hay chậm).
Nếu đặt $\Delta s$ là độ biến thiên tọa độ của chất điểm trong khoảng thời gian $\Delta t$,
thì vận tốc là $v = \frac{\Delta s}{\Delta t}$.

Vận tốc tức thời đặc trưng cho sự nhanh chậm của chuyển động.
Ta không thể khảo sát vận tốc tại 1 điểm vì chất điểm chuyển động
chứ không đứng yên. Do đó một ý tưởng đơn giản là chúng ta khảo sát
trong một khoảng thời gian cực nhỏ, khi $\Delta t \to 0$, khi đó
vận tốc gần như đúng tại một thời điểm nên gọi là vận tốc tức thời.

Do đó ta có hàm số $v = s'(t)$ biểu diễn vận tốc theo thời gian, 
với $s(t)$ là hàm số biểu diễn chuyển động của chất điểm theo thời gian.

\section{Cực trị}

Đầu tiên chúng ta cần một định lý về tính đơn điệu của hàm số
khả vi.

\begin{theorem}
    Xét hàm số $f(x)$ khả vi trên khoảng $(a, b)$. Nếu $f'(x) > 0$ 
    với mọi $x \in (a, b)$ thì $f(x)$ đồng biến trên $(a, b)$.
\end{theorem}

Tương tự, $f'(x) < 0$ với mọi $x \in (a, b)$ thì $f(x)$ nghịch biến trên
$(a, b)$.

\begin{definition}[Cực tiểu của hàm số]
    Xét hàm số $f(x)$ liên tục trên khoảng $(a, b)$. Điểm $(x_0, f(x_0))$ được
    gọi là \textbf{cực tiểu} của hàm số $f(x)$ nếu tồn tại một lân cận $U$
    chứa $x_0$ nằm trong khoảng $(a, b)$ sao cho với mọi $x \in U$ thì $f(x) \geq f(x_0)$.
\end{definition}

\begin{definition}[Cực đại của hàm số]
    Xét hàm số $f(x)$ liên tục trên khoảng $(a, b)$. Điểm $(x_0, f(x_0))$ được
    gọi là \textbf{cực đại} của hàm số $f(x)$ nếu tồn tại một lân cận $U$
    chứa $x_0$ nằm trong khoảng $(a, b)$ sao cho với mọi $x \in U$ thì $f(x) \leq f(x_0)$.
\end{definition}

Theo định nghĩa cực tiểu thì chỉ cần tồn tại lân cận chứa $x_0$
mà $f(x) \geq f(x_0)$ thì điểm đó là cực tiểu. Như vậy một hàm số có 
thể có nhiều cực tiểu, tương tự cũng có thể có nhiều cực đại.

Lưu ý rằng cực đại và cực tiểu không phải điểm chỉ giá trị lớn nhất
hay giá trị nhỏ nhất của hàm số. Nó chỉ lớn nhất hoặc nhỏ nhất trong
vùng lân cận đó theo định nghĩa, nên người ta còn gọi là cực trị địa phương.

%Xét hàm số $f(x)$ khả vi trên khoảng $(a, b)$. Gọi $f'(x)$ là
%đạo hàm của hàm số $f(x)$ trên $(a, b)$. Khi đó điểm $x_0 \in (a, b)$ được gọi là
%\begin{itemize}
 %   \item Cực tiểu nếu $f'(x)$ đổi chiều từ âm sang dương khi đi qua $x_0$
 %   \item Cực đại nếu $f'(x)$ đổi chiều từ dương sang âm khi đi qua $x_0$
%\end{itemize}

